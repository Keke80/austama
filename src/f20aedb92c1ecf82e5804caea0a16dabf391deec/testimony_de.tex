Lyndsay Morton spricht über ihre Tochter, am 27. Juni 2021:

Am 26. Juni um 10 Uhr hatte meine 19-jährige Tochter ihre erste Impfung. Sie
fühlt sich gut und wir gehen tagsüber einkaufen. Auf dem Heimweg gegen 16 Uhr,
fühlt sie sich, als on sie reisekrank wäre.

Um 16 h 15 Uhr bricht sie zusammen. Wir versuchen sie aufzurichten damit wir sie
schnell nach Hause bringen können. Aber sie bricht wieder zusammen. Wir haben
ihren Vater angerufen, damit er mit dem Auto kommt, um sie nach Hause zu
fahren. Sie bricht beim Aussteigen aus dem Auto auf der Straße vorm Haus wieder
zusammen. Dann bekommt sie ein Kribbeln in Armen und Beinen, Schmerzen in der
linken unteren Lunge und ein Engegefühl im Hals. Wir rufen einen
Krankenwagen. Der Notruf teilte uns mit, dass sie schnell behandelt werden
müsse, aber sie haben viel zu tun und es würde schneller gehen, wenn ich sie
selbst hinbringe.

Um 17h30 Uhr kommen wir im Whiston Hospital (Liverpool, UK) an, können sie aber
nicht aus dem Auto heben. Aber da die Krankenschwestern zu beschäftigt sind,
bemühen wir uns selbst, sie in einen Rollstuhl zu setzen. Wir gehen hinein und
erklären, was der Notruf gesagt hat, und man sagt uns, wir sollen im Warteraum
mit über 30 anderen Leuten warten. Sie bekommt im Stuhl sitzend Krämpfe und die
Schmerzen in Lunge und Bein nehmen zu.

Zwei Stunden später sitzt sie immer noch in diesem Zustand da, und ich
beschließe, ein Taxi zu rufen, das uns zum Royal Hospital bringt. Als das Taxi
vorfährt, versuche ich, ihr aus dem Rollstuhl zu helfen, aber sie bricht wieder
auf dem Boden zusammen. Keine einzige Krankenschwester kam, um ihr zu helfen,
aber andere Patienten kamen herbei, um zu helfen, sie in das Taxi zu setzen.

20.00 Uhr - wir kommen im Royal A\&E an, ich gehe hinein und frage nach einem
Rollstuhl, man sagt mir: “Wenn Sie keinen sehen, gibt es auch keinen”, und ich
gehe wieder nach draußen, wo meine Tochter auf dem Rücksitz eines Taxis
zusammengesackt und und mit Krämpfen liegt. Der Taxifahrer sucht draußen einen
Rollstuhl und hilft mir, sie ins Krankenhaus zu bringen.

Die Krankenschwester nahm sie innerhalb von 5 Minuten auf, und während sie sie
in den Stuhl schob, sagte sie ihr, sie glaube ihr nicht, dass sie nicht laufen
könne und dass sie nicht darauf (auf den Impfstoff) reagiere, da dies noch nie
passiert sei. Man sagt mir, ich soll gehen und jemand werde mich in etwa 3
Stunden kontaktieren. Eine 19-Jährige, die Angst hat allein lassen? Aber ich
respektiere die Vorschriften und gehe.

Sie wird im Hauptempfangsraum zurückgelassen und dann in eine Nebenraum verlegt,
da man nicht will, dass sie sich schämt, wenn sie zittert (für mich ist sie zu
diesem Zeitpunkt eher ein Versteck). Fünf Stunden liegt sie auf diesem Bett
allein. Es geht ihr nicht besser, ihr ist kalt, es gibt keine Decke, kein
Essen. Letztendlich schafft sie es, eine Krankenschwester um Schmerzmittel zu
bitten, da sie nun Kopfschmerzen hat!

Um 0h30 Uhr nachts, 8 Stunden nach ihrem ersten Zusammenbruch, kommt ein Arzt,
der ihr sagt, dass es sich um Nebenwirkungen (des Impfstoffs) handelt, die 1-2
Tage andauern werden, und wenn sie länger andauern, muss sie wiederkommen. Sie
wird entlassen und muss um 1 Uhr morgens in der Samstagnacht alleine vor dem
Royal Hospital warten, schwach und frierend, bis wir sie abholen können! Wir
haben Videoaufnahmen, die zeigen, wie sie 6 Stunden lang liegen gelassen wurde,
einschließlich von ihren Krämpfen!

BITTE DENKEN SIE ZWEIMAL darüber nach, ob Sie Ihre Kinder impfen lassen wollen!

Brooklyn erzählt ihre eigene Geschichte:

“Leider hatte ich eine Reaktion auf den Impfstoff. Ich hätte nie gedacht, dass
ich jemand sein würde, der von den Nebenwirkungen des Impfstoffes betroffen sein
würde, wenn man bedenkt, dass er von unserer Regierung in einem so positiven
Licht angepriesen wird.

Dies war eines der schrecklichsten Dinge, die ich je in meinem Leben erlebt
habe. Wegen der Ungewissheit über die Diagnose gingen mir viele Fragen durch den
Kopf: Wird das für immer so bleiben? Was geschieht mit mir? Warum passiert das
mit mir? Erlebe ich ein Blutgerinnsel oder einen Anfall? Während mir diese
Fragen durch den Kopf gingen, befand ich mich in einem Raum, der vom Rest des
Wartezimmers abgeschirmt war, weil man dachte, ich würde mich wegen meiner
Krämpfe “zu sehr schämen”. In Wirklichkeit wurde ich vom Rest der Welt wegen der
tatsächlichen Nebenwirkung des Impfstoffs versteckt! Ich habe überlegt, ob ich
mich impfen lassen soll, wegen der unbekannten Wirkungen und auch wegen der
Geschichten, die man von anderen darüber hört. Ich verurteile niemanden,der sich
impfen lässt, aber ich möchte die Menschen sensibilisieren und sie sollten sich
fragen, ob es das Richtige für sie ist Ich bin 19 Jahre alt und vollkommen
gesund. Ich habe keine grundlegenden medizinischen Probleme. Warum also sollte
mich das so beeinträchtigen?”
