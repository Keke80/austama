\testimony{Brooklyn Neal}
          {Liverpool, Royaume-Uni}
          {19 ans}
          {Pfizer/BioNTech}
          {26 juin 2021}
          {Convulsions, douleurs dans les poumons et les jambes
          }
          {picture.jpg}
          {https://nomoresilence.world/pfizer-biontech/brooklyn-neal-aged-19-pfizer-adverse-reaction/}
          {

\normalsize

Lyndsay Morton, la mère de Brooklyn, raconte l'histoire de sa fille le 27 juin
2021:

A 10 heures le 26 juin, ma fille de 19 ans a reçu sa première dose de
vaccin. Elle allait parfaitement bien et nous sommes allées faire du shopping
pour la journée. En rentrant à la maison vers 16 heures, elle a commencé à se
sentir mal dans les transports, comme si elle avait le mal des transports.

à 16h15, elle s'est effondrée. On l'a relevée, essayant de la faire marcher
jusque la maison, puis elle s'est effondrée de nouveau. Nous avons dû appeler
son père pour qu'il vienne la chercher et la mettre dans sa voiture, il a
conduit 2 minutes, ils sont sorti de la voiture et elle s'est évanouie de
nouveau, sur la route devant sa maison.

A ce moment, elle ressent des aiguilles dans les bras et les jambes, une douleur
dans le bas de son poumon gauche, et sa gorge est serrée. On a appelé les
urgences, qui nous ont dit qu'elle devait vite être amenée à l'hopital, mais
qu'ils étaient occupés et que je pouvais suûrement l'emmener là-bas plus
rapidement moi-même.

A 17h30, nous sommes arrivé à l'hopital Whiston (à Liverpool) mais nous n'avions
pas réussi à la sortir de la voiture, et il n'y avait personne pour aider, car
les infirmières étaient trop occupées. Avec difficulté, nous parvenons à la
mettre en fauteuil roulant, l'emmenant à l'intérieur en expliquant ce qu'ont dit
les urgences au téléphone, et ils nous ont dit d'aller dans la salle d'attente,
où se trouvaient plus de 30 autres personnes alors qu'elle était entrain de
convulser dans la chaise avec des douleurs grandissantes dans ses poumons et ses
jambes.

Deux heures plus tard, nous sommes toujours assis avec elle dans les mêmes
conditions, alors j'ai décidé d'appeler un taxi pour nous emmener au Royal
Hospital. Alors que le taxi arrive, j'essaie de l'aider à sortir de sa chaise
roulante, mais elle s'effondre encore sur le sol. Aucun soignant n'est venu nous
aider, mais d'autres patients sont venus pour l'aider à l'asseoir dans le taxi.

20h - nous arrivons au Royal A\&E, je vais a l'intérieur et je demande pour une
chaise roulante, ils me répondent “si vous n'en voyez pas, c'est qu'il n'y en a
pas”. Je ressors pour retrouver ma fille à l'arrière d'un taxi se tordant
entrain de convulser. Le conducteur de taxi est sorti et a trouvé une chaise
roulante à l'extérieur, puis m'a aidé à l'amener dans l'hopital.

Une infirmière l'a emmené dans les 5 minutes, et alors qu'elle pousse sa chaise
roulante, elle lui dit qu'elle ne la croit pas quand elle dit qu'elle ne peut
pas marcher, et qu'elle ne croit pas qu'elle ait une réaction au vaccin, comme
si ça n'était jamais arrivé. On me dit de partir et que quelqu'un me
contacterait dans environ 3 heures. Avec le coeur lourd je laisse ma fille de 19
ans seule, terrifiée, mais je respecte leurs règles et je pars.

Elle est restée dans la salle de réception principale, puis a été bougée dans
une salle latérale car ils ont dit qu'ils ne voulaient pas qu'elle se sente
génée de convulser (pour moi, à ce stade, ils voulaient juste la cacher).

Elle a été laissée seule dans ce lit pendant 5 heures, sa condition ne
m'améliorait pas, elle était dans le froid, pas de couverture, pas de
nourriture, et a finalement réussi à voir une infirmière qui lui a donné des
antidouleurs puisqu'à présent elle avait développé des maux de tête.

A 00:30 du matin, 8 heures après qu'elle se soit effondrée pour la première
fois, elle voit un docteur qui lui dit qu'il s'agit d'un effet secondaire lié au
vaccin qui durera 1 ou 2 jours, et que si ça durait plus longtemps il faudrait
revenir. Elle a été amenée à l'extérieur et a dû attendre en dehors de l'hopital
seule à 1 heure du matin un samedi, fébrile et gelée avant qu'on ne puisse la
rechercher!

Nous avons des vidéos montrant comment elle a été abandonnée pendant 6 heures,
notamment des vidéos de ses convulsions!

S'il vous plaît, réfléchissez à deux fois avant de laisser vos enfants prendre
ce vaccin!

Brooklyn raconte sa propre histoire:

“Malheureusement, j'ai eu une réaction au vaccin. Je n'ai jamais pensé que
j'aurais pu être l'une des personnes affectées par le vaccin, puisque ça avait
l'air d'être si positif et sûr pour notre gouvernement.

Ca a été l'une des choses les plus terrifiantes que j'ai vécu de toute ma
vie. Pendant tout ce temps, des questions envahissaient mon esprit… est-ce que
ça va durer éternellement? Qu'est-ce qu'il m'arrive? Pourquoi est-ce que ça
m'arrive? Suis-je victime de caillots sanguins ou d'une crise?

Alors que ces questions allaient et venaient dans ma tête, j'étais dans une
pièce cachée de la salle d'attente, parce qu'ils ont cru que j'allais être “trop
embarrassée” par mes convulsions. Alors qu'en vérité, ils me cachaient du reste
du monde parce que j'avais des effets indésirable du vaccin!

J'ai débattu avant de le recevoir, avec les effets inconnus et aussi les
histoires que vous voyez des autres à ce sujet.

Je ne suis personne pour juger la décision de quelqu'un sur oui ou non recevoir
le vaccin, mais je veux que les gens soient au courant et qu'ils se posent des
questions sur si c'est la bonne chose à faire pour eux.

J'ai 19 ans et je suis en parfaite santé. Je n'ai jamais eu d'antécédants
médicaux. Alors pourquoi ça m'arrive?”

}
