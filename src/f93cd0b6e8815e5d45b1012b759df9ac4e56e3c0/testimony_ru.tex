Мой партнер получил вторую дозу AstraZeneca 6 июля 2021 года. Незадолго до этого
он был в хорошей форме, был здоров и не принимал никаких лекарств.

Он подумал, что заболел, так как него появился понос, и 16 июля он вернулся с
работы пораньше. 17-го он все еще плохо себя чувствовал, и в ту ночь, около 11
часов вечера рухнул на пол кухни. У него случился клонико-тонический припадок.

Скорая помощь приехала через два с половиной часа после вызова. За это время у
него случилось ещё 4 эпилептических припадка. Это произошло в ванной, где я
смогла увидеть, что у него была не диарея, а, на самом деле, кровотечение со
сгустками крови.

Его повезли в больницу одного в машине, ввиду новых правил из-за коронавируса. Я
приехала туда через 4 часа, и мне сказали, что у него кровоизлияние в желудок и
в мозг.

Ему стало немного лучше, но это только после МРТ, сканнера, колоноскопии и
эндоскопии, двух переливаний крови и донорских тромбоцитов. Мы искали ответ на
наш вопрос: может ли это быть связано с вакцинацией? Но медицинский персонал
ответил пренебрежительно-неопределённо.

Он не мог стоять, у него была слабость в левой руке и ноге, а также опухоль,
появившаяся в течение ночи. «У него не было инсульта,“ - сказали нам. “Просто он
принимает очень большие дозы Кеппры, от чего он утомляется и становится очень
раздражительным.”

Сейчас, после выписки, он не может работать и получает только установленное
законом пособие по болезни. Я взяла отпуск, чтобы присмотреть за ним.

Отвратительно, что они не хотят давать нам прямой ответ. Я никогда никому не
скажу не делать вакцину, но меня злит, что никто из них не признаёт, что
существует настоящая проблема. Как они могут иметь правильную статистику, если
они даже не учитывают инвалидов и мертвых?

