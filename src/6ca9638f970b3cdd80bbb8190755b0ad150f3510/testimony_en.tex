Tenley, a creative, athletic, and vibrant 22-year-old, had just graduated from
college and started working at her first job in her new career as a Rehab
Therapist Assistant, helping the elderly in a Long Term Care facility last
summer.

Then in late December 2020, she was given an injection of the Moderna Vaccine in
one of the first waves for health care workers in British Columbia due to the
nature of her work.

She experienced some nausea and other symptoms immediately following the
shot. They resolved fairly quickly, except for a headache and fatigue that
persisted.

On January 18, 2021, about two and a half weeks after the Vaccine, her legs felt
unstable and wobbly while at work and she was encouraged to leave her work shift
early. The next morning she woke unable to stand or walk on her own without
additional support, and ultimately collapsed at home. She was immediately taken
to the ER.

Tenley then spent two weeks in hospital, and has since been seen by
hospitalists, neurologists, a psychiatrist, has had two MRIs, and a lumbar
puncture. Since her release from the hospital, she has had very limited medical
support or direction as to treatment or physical therapy care as there is no
current diagnosis or prognosis.

No health care workers or doctors have ruled out the Vaccine as the cause, and
unfortunately, none have definitively documented that it is directly
related. She is in medical limbo-land.

Seriously, just think about it all for just a moment. The chances are extremely
unlikely for a healthy and active 22-year-old to wake up one day and not be able
to walk!

Tenley is now under a physiotherapist who has the extensive neurological
experience and is working with her to incorporate more functional movement and
recommended exercises into her daily program.

None of Tenley’s medical or rehabilitation costs have been met by the government
or insurance.


