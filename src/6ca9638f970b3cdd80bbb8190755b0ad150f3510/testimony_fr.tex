\testimony{Tenley  Csolle}
{British Columbia, Canada}
{22 ans}
{Moderna}
{30 Decembre 2020}
{Nausées, maux de têtes, fatigue, jambes instables, problème neurologique}
{picture.jpg}
{https://nomoresilence.world/moderna/tenley-csolle-moderna-severe-adverse-reaction/}
{

Tenley, une jeune femme de 22 ans créative, athlétique et dynamique, elle venait
d'obtenir son diplôme universitaire et avait commencé à travailler pour son
premier emploi dans sa nouvelle carrière d'assistante thérapeute en rééducation,
aidant les personnes âgées dans un établissement de soins de longue durée l'été
dernier.

Puis, fin décembre 2020, elle a reçu une injection du vaccin Moderna dans l'une
des premières vagues pour les travailleurs de la santé en Colombie-Britannique
en raison de la nature de son travail.

Elle en est ressortis avec quelques nausées et quelques autres symptômes
immédiatement après l'injection. Ceux-ci se sont résorbés assez rapidement, à
l'exception d'un mal de tête et d'une fatigue persistante.

Le 18 janvier 2021, soit environ deux semaines et demie après le vaccin, ses
jambes sont devenues instables et chancelantes au travail, on l'a encouragée à
quitter son poste plus tôt. lendemain matin, elle s'est réveillée incapable de
se tenir debout ou de marcher seule sans soutien supplémentaire, et s'est
finalement effondrée à la maison. Elle a été immédiatement emmenée aux urgences.

Tenley a ensuite passé deux semaines à l'hôpital et a depuis été suivie par des
hospitaliers, des neurologues, un psychiatre, a subi deux IRM et une ponction
lombaire. Depuis sa sortie de l'hôpital, elle a bénéficié d'un soutien médical
très limité ou d'une orientation en matière de traitement ou de soins de
physiothérapie, car il n'existe aucun diagnostic ni pronostic actuel.

Aucun travailleur de la santé ou médecin n'a exclu le vaccin comme cause, et
malheureusement, aucun n'a documenté de manière définitive si il est directement
lié. Elle est tombée dans les méandres de la médecine.

Sérieusement, réfléchisez-y un instant. Il est extrêmement improbable qu'une
jeune femme de 22 ans, active et en bonne santé, se réveille un jour sans
pouvoir marcher !

Tenley est maintenant suivie par un physiothérapeute qui a une grande expérience
neurologique et qui travaille avec elle pour ré-intégrer des mouvements plus
fonctionnels et des exercices recommandés dans son programme quotidien.

Aucun des frais médicaux ou de rééducation de Tenley n'a été pris en charge par
le gouvernement ou les assurances.

}
