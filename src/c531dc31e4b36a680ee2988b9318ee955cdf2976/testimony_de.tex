Holly, 34 Jahre alt aus Leeds, Großbritannien, bekam am 7. Mai 2021 den
Impfstoff von Pfizer. Sie bekam eine Hirnblutung, die eine Hirnoperation zur
Folge hatte und sie verfiel in ein Koma.

Holly war Erzieherin in einer Grundschule und arbeitete mit 2-jährigen
Kindern. Sie ist eine wunderbare, gutherzige Seele und dachte, dass die Impfung
die Menschen in ihrer Umgebung schützen würde, insbesondere bei der Arbeit mit
Kindern.

Fünf Wochen nach der Impfung bekam Holly Kopfschmerzen und Übelkeit, und zwei
Wochen später, am 18. Juli, brach Holly zusammen und wurde ins Leeds General
Infirmary eingeliefert. Das Krankenhaus bestätigte nach einer MRT-Untersuchung,
dass Holly vier Blutgerinnsel in der rechten Gehirnhälfte hatte.

Holly wurde sofort einer neurochirurgischen Operation unterzogen, um die
Gerinnsel zu entfernen, und es wurde eine Drainage in ihren Schädel gelegt, um
die, durch die Hirnschwellung entstandene Flüssigkeit zu entfernen.

Nach der Operation lag Holly im Koma und blieb auf der Intensivstation. Holly
war nicht in der Lage, selbständig zu atmen, und wurde intubiert. Inzwischen hat
sich ihr Zustand jedoch gebessert, und sie kann nun ohne fremde Hilfe atmen,
obwohl sie einem Luftröhrenschnitt hatte.

Zum Entsetzen der Familie verweigerte das Krankenhaus anfangs jegliche Besuche
bei Holly, die länger als eine halbe Stunde pro Woche dauerten, mit der
Begründung, dies sei eine behördliche Anordnung aufgrund des Covid!

Wir mussten dringend Geld für die Anwaltskosten der Familie aufbringen, um gegen
die Entscheidung des Krankenhauses anzukämpfen, damit die Familie sie auf der
Station besuchen konnte. Dieses Verfahren wurde am 17. August von No More
Silence eingeleitet und koordiniert. Die Anwaltsgebühren wurden von No More
Silence übernommen, obwohl die Kosten für einen Rechtsstreit in diesem Fall sehr
hoch sein werden.

Als die Familie zu Besuch kam (ein halbstündiger Besuch), drückte Holly die Hand
ihres Mannes und ihrer Schwester, öffnete ihre Augen und streckte auf Kommando
die Zunge heraus. Dies hatte das Krankenhauspersonal vor dem Besuch der Familie
nicht erreicht und es zeigt, dass die Interaktion der Familie bei Hollys
Genesung absolut notwendig war.
