Holly, 34 ans, de Leeds (Royaume-Uni), a pris le vaccin Pfizer le 7 mai 2021, ce
qui l'a plongée dans le coma en raison d'une hémorragie cérébrale qui a
nécessité une opération du cerveau.

Holly était infirmière dans une école primaire et travaillait avec des enfants
de 2 ans. C'est une belle âme au grand cœur et elle pensait qu'en se faisant
vacciner, elle protégerait son entourage, surtout lorsqu'elle travaille avec des
enfants.

Cinq semaines après le vaccin, Holly a commencé à présenter des maux de tête et
des maladies débilitantes, puis deux semaines plus tard, le 18 juillet, Holly
s'est effondrée et a été admise à l'infirmerie générale de Leeds. L'hôpital a
confirmé, après avoir fait une IRM, qu'Holly avait 4 caillots de sang dans le
côté droit de son cerveau.

Holly a été envoyée immédiatement en neurochirurgie cérébrale pour enlever les
caillots et un drain a été inséré dans son crâne pour enlever le liquide
accumulé par le gonflement du cerveau.

Après l'opération, Holly est restée dans le coma et aux soins intensifs. Holly
était incapable de respirer par elle-même et a été intubée, bien que nous ayons
constaté des améliorations et qu'elle puisse maintenant respirer sans aide, bien
qu'elle ait été équipée d'une trachéotomie.

À l'horreur de la famille, l'hôpital a d'abord refusé toute visite d'Holly autre
qu'une demi-heure par semaine, expliquant qu'il s'agissait d'une directive
gouvernementale due au Covid!

Il était urgent de collecter des fonds pour payer les frais d'avocat de la
famille afin qu'elle puisse lutter contre la décision de l'hôpital et obtenir la
présence de la famille dans le service. Cette procédure juridique a été lancée
et coordonnée le 17 août par No More Silence. Les frais d'instruction ont été
payés par No More Silence, bien que pour combattre cette affaire légalement, les
coûts seront beaucoup plus élevés.

Lors de la visite de la famille (visite d'une demi-heure), Holly a serré la main
de son mari et de sa sœur, a ouvert les yeux et a tiré la langue sur
commande. Cela n'avait pas été vu par le personnel de l'hôpital avant la visite
de la famille et cela montre que l'interaction des familles dans le
rétablissement d'Holly était absolument nécessaire.

