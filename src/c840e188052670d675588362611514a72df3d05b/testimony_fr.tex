Antoine Méchin, un triathlète professionnel français, a souffert d'une embolie
pulmonaire immédiatement après avoir reçu le vaccin Moderna.

Il écrit :

``Saison sportive terminée, suite à une embolie pulmonaire due au vaccin
Covid-19 - 1ère dose Moderna fin juin 2021.

Deux semaines plus tard, alors que je suis dans les Pyrénées pour un
entraînement avec le compère d'Arnaud Guilloux, je commence à ressentir une
oppression.

Je participe alors à un trail sur le chemin du retour et sur cette course, je
sens que je suis rapidement essoufflé et une douleur dans le bras gauche, le dos
et le cœur.

Le lendemain 18 juillet, mon médecin me conseille d'aller faire des examens à
l'hôpital de Saintes. Pas de soucis détectés, ils me disent que c'est
certainement lié au stress, et à la fatigue.

Je pars donc aux Etats-Unis pour faire l'IM Lake Placid. Quand je ne fais pas de
cardio trop élevé, je ne ressens pas grand chose et donc la course se passe bien
pour moi, sans connaître de désagrément notable.

Après ma pause sportive de quelques jours suivant l'Ironman, je reprends
progressivement sur des intensités faibles, je sens que je ne suis pas en forme
mais pas de douleur spécifique. Je fais ensuite ma 2ième dose de vaccin, je peux
donc nager en piscine.

Je me sens bridé sur les efforts à haute intensité, j'ai parfois des gênes dans
la poitrine et le dos, mais je ne m'inquiète pas. Je fais le Duathlon D1,
sachant que je suis bridé, et le lendemain de la course je ressens plus cette
oppression thoracique. Je fais du vélo le lundi avec @pacometl pour lui montrer
notre belle côte charentaise, et finis en solo avec 5/10ème bien pressé où je
sens que j'ai de la force mais il y a une inquiétude au niveau du souffle.

Je parle à mon médecin sur l'insistance d'un ami sponsor jardinsdesaintonge qui
ne me trouve pas bien, et je vais à Bordeaux pour un rendez-vous en cardiologie
aujourd'hui. J'explique mes symptômes, et le cardio ultra compétent de
medical.stadium m'explique le nombre croissant de réaction post vaccinale de ce
type et diffuse en urgence une scintigraphie.

Je suis maintenant sous traitement et espère retrouver mes capacités pulmonaires
dans 3/6/9/12 mois ? En attente de repos et de faible intensité pendant
plusieurs mois.

Triste réalité, triste mesure, triste gouvernement !
