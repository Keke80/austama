\testimony{Aiden Jo}
          {Georgie, USA}
          {14 years old}
          {Pfizer/BioNTech}
          {12th May 2021}
          {Myocarditis}
          {picture.jpg}
          {https://nomoresilence.world/pfizer-biontech/aiden-jo-pfizer-severe-adverse-reaction/}
          {

\normalsize

Emily Jo, Aiden’s mother, said “before her son got the vaccine, she was led to
believe his chance of suffering an adverse reaction was one-in-a-million.”

Aiden received his first dose of Pfizer on May 12th. He had no history of COVID
or pre-existing conditions except for asthma.

On June 10th, several days after his second dose, Aiden woke his mother up at
4:30 a.m. because his chest hurt and he couldn’t breathe.

Jo said she was aware of the potential side effect of heart inflammation, but
the Centers for Disease Control and Prevention (CDC) said it was very rare and
mild. “What they didn’t explain is that mild means hospital care and follow-up
care indefinitely,”

Jo took her son to an emergency room in Atlanta, where the CDC has its
headquarters. The ER doctor first asked whether Aiden had his vaccine, then
tested his troponin level to see if his heart muscle had been damaged. Doctors
also ran an EKG.

After both tests came back as abnormal, Aiden was admitted to the acute cardiac
unit.

The first night Aiden stayed in the hospital his troponin number quadrupled
overnight. Aiden’s mother said the cardiologist reported it to the CDC’s Vaccine
Adverse Events Reporting System (VAERS I.D. 1396660), and was in contact with
the CDC multiple times during his four-day hospital stay.

“They determined that yes, he met the criteria of having post-vaccine
myocarditis,” Jo said.

“The biggest problem is they [CDC] are not explaining what mild myocarditis
means,” Jo said. “Aiden’s cardiologist told us no case of myocarditis is ‘mild.’
That’s like saying a heart attack is mild.”

Jo said her son gets very tired easily and his recovery will be a long process.

“He can’t do any physical activity, no recess, no PE, he has to have more time
to go between classes until he has a cardiopulmonary stress test that shows his
heart can handle the stress,” she said

Jo, who said she’s been the target of animosity from ‘anti-vaxxers’ for
vaccinating her son, and now from ‘pro-vaxxers’ for telling her son’s story of
vaccine injury, stated that the guilt is eating her up.

“I was one of those jerks who was like, ‘Oh it’s your fault. You’re the reason
everybody needs to get vaccinated,’ so this has flipped everything for me upside
down,” Jo said.

Jo said all her kids are fully vaccinated and she was one of the most trusting
advocates of the CDC and American Academy of Pediatrics (APA) — until her son
experienced his vaccine injury.

“They’ve lost me and they’re going to lose a lot of people,” she said. “When you
lose trust in public health we have a big problem. They’ve lost one of their
biggest advocates and I don’t think it can ever be earned back.”

Jo said, if you were to ask her today whether she would vaccinate her children,
she would say “no,” because the vaccine is not effective and kids are at a lower
risk.

Jo said prior to her son having the reaction, she was not aware vaccine makers
were exempt from liability. She thinks parents need to know there is little
recourse should their child get injured by a vaccine.

She said: “I think another thing parents need to understand is that myocarditis
is not covered under the National Vaccine Injury Compensation Program, and the
Countermeasures Injury Compensation Program only covers if you’re incapacitated,
wheel-chair bound or dead. We have incurred thousands and thousands of dollars
in medical bills. We have insurance but they don’t pay everything. And it does
not account for tests down the road that we still have to get.”

Jo said she knows the financial part is not the main concern, but she’s a
teacher and doesn’t have thousands of dollars sitting around.

“I don’t feel I should have to pay for doing what I was told to do by the
government,” Jo said. “Hey we are all in this together and then you get a
vaccine injury and you’re just completely ignored — and not just ignored but
beaten up from both sides.

Jo doesn’t understand how the government can mandate something when there is no
culpability — and she’s questioning why there is no culpability.

“Why are these children just shoved into a corner as collateral damage as if it
doesn’t matter?” she asked.

Jo added: “I’ve known many, many children who’ve gotten COVID. This is
anecdotal. I know there are serious cases but anecdotally, I have seen 15-20
kids who have had COVID. They had the sniffles and my kid is the one who ended
up in the hospital because of the vaccine.”

}
