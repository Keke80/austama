Je m'appelle Liz. J'ai 36 ans et voici mon histoire.

Le 22 mars 2021, j'ai reçu la première dose du vaccin Moderna. J'avais 35 ans et
je n'ai plus jamais été la même depuis !

Le lendemain, moins de 24 heures plus tard, j'ai commencé à ressentir de fortes
douleurs dans la poitrine. Au début, je n'y ai pas prêté attention ; j'essayais
de les ignorer en pensant qu'elles disparaîtraient. Il est devenu de plus en
plus difficile de l'ignorer, car il s'agissait maintenant d'une douleur
thoracique constante accompagnée de douleurs aiguës.

Le 3 avril, j'ai senti mon cœur s'emballer. Mon rythme cardiaque est monté à 173
BPM, puis mon bras gauche s'est engourdi et j'ai senti ma main gauche
picoter. J'avais l'impression de ne pas pouvoir respirer. J'ai cru que je
faisais une crise cardiaque. Nous avons appelé le 911 et sommes allés aux
urgences.

Le médecin m'a posé des questions et je lui ai dit que j'avais toujours été une
personne en bonne santé, sans problèmes. La seule chose nouvelle que j'avais
faite était de recevoir le vaccin Moderna. Elle m'a alors regardée et m'a dit :
“Vous faites une réaction au vaccin. Je vous déconseille de prendre la deuxième
dose.” Elle a dit que mon coeur allait bien, qu'il n'y avait pas de crise
cardiaque et que tous les tests étaient bons. Si seulement je savais que ce
n'était que le début… mais les choses allaient devenir bien pires !

Quatre jours plus tard, j'ai consulté mon médecin qui m'a dit que c'était
probablement des reflux acides et de l'anxiété. Mon médecin m'a donné des
médicaments pour cela, mais elle ne croyait pas que c'était le vaccin. Je
connais bien mon corps ; je savais que quelque chose n'allait pas du tout. Plus
tard dans la soirée, mes symptômes se sont reproduits. Cette fois, mon visage
s'est engourdi, j'ai eu des picotements et mon cœur s'est emballé à nouveau. Je
me suis de nouveau rendue aux urgences et, cette fois, j'ai été admise pendant
trois jours. J'ai pu voir un cardiologue et j'ai passé un échocardiogramme et
d'autres tests. Tout est revenu clair. Je faisais maintenant des crises
d'angoisse alors que je n'en avais jamais eu auparavant.

Après ma sortie de l'hôpital, j'étais si faible, fatiguée et souffrante. J'ai dû
quitter ma maison et retourner vivre dans ma famille pour qu'on puisse s'occuper
de moi, car je ne pouvais pas rester seule. Je n'étais plus indépendante.

Je prends maintenant un médicament pour abaisser mon rythme cardiaque, mais
personne n'a pu me dire pourquoi cela se produisait. Mon anxiété était telle que
je ne pouvais pas dormir dans le noir, je ne pouvais pas avoir la porte fermée,
les volets devaient être ouverts. N'importe quoi pouvait déclencher une crise
d'angoisse : un bruit fort, trop de mouvement, trop de gens qui parlent, le
goût, l'odeur, etc. Tout était tellement écrasant pour moi que je ne pouvais
même pas faire des choses simples comme répondre à un SMS ou passer un appel. Je
n'avais jamais rien vécu de tel auparavant. J'avais l'impression que ce n'était
pas mon corps, que je n'avais aucun contrôle sur lui. À ce stade, chaque jour
apportait un nouveau symptôme.

J'ai commencé à ressentir de fortes douleurs dans les ovaires et je me suis
rendue aux soins d'urgence où l'on m'a fait passer de nombreux tests qui se sont
avérés concluants. Entre-temps, j'ai consulté un nouveau cardiologue qui a
accepté de faire d'autres tests et de m'installer un moniteur holter. Je
n'arrivais pas à comprendre ce qui arrivait à mon corps.

Puis d'autres nouveaux symptômes sont apparus : tremblements, secousses,
tremblements, nervosité, faiblesse musculaire, spasmes musculaires, sensation de
brûlure, douleurs fulgurantes, vertiges, étourdissements, évanouissement proche,
essoufflement, oppression thoracique, rythme cardiaque élevé, insomnie,
difficulté à marcher, le côté droit de mon corps ressentant une douleur bien
pire.

Mes symptômes comprenaient même des crises de panique, des migraines, des
difficultés à articuler, des problèmes de mémoire, un brouillard cérébral
important, une fatigue extrême, des problèmes de vision, des acouphènes, des
douleurs à l'oreille droite, une perte de cheveux, une difficulté à former des
phrases ou à trouver des mots, et une incapacité à me concentrer. Je pensais que
j'étais en train de mourir.

Certains jours, j'étais si mal en point que ma famille devait me nourrir à la
cuillère, m'aider à m'habiller et à me doucher.

Je suis retournée aux urgences, où j'ai subi de nombreux autres tests, mais
toujours rien, aucune réponse à la question de savoir pourquoi cela se
produisait. À ce stade, j'étais tellement découragée que j'avais l'impression
que je n'allais pas obtenir d'aide. Des médecins m'ont demandé si je prenais de
la drogue. Ils m'ont dit : “C'est psychologique, c'est dans ta tête, une fois
que tu auras arrangé ton esprit, tu te sentiras mieux, c'est juste de l'anxiété,
c'est juste du stress, tu dois juste être plus positive”.

C'est triste de voir comment ils vous font croire que vous êtes fou. On m'a
ensuite diagnostiqué une fibromyalgie et le médecin a dit que c'était juste une
coïncidence que tout cela se produise après le vaccin, mais que le vaccin n'en
était pas la cause.

Après avoir consulté une spécialiste des maladies neuromusculaires, elle a
confirmé que je n'étais pas la seule à vivre cela. Ils avaient des cas
similaires au mien, qui avaient été vaccinés plus tôt en décembre et en janvier
et qui présentaient encore des symptômes (c'était en juin, alors que je n'en
étais qu'au troisième mois).

Je voyais alors une infirmière praticienne en psychiatrie, tous les médicaments
contre l'anxiété et la dépression aggravant considérablement la situation. Je
subissais maintenant tous les effets secondaires de ces médicaments. Je ne
prenais plus aucun médicament, j'essayais celui-ci ou celui-là, peut-être que
celui-là vous aidera. Rien ne fonctionnait pour moi.

Puis, un jour, j'ai commencé à avoir un gonflement du visage, de la gorge et de
tout le côté droit de mon corps. Ma jambe droite était très enflée et la douleur
était extrême. Je suis retournée aux urgences. Ils ont fait une échographie pour
voir s'il y avait un caillot de sang et tout est revenu à la normale. Toujours
pas de réponse.

Après que tous mes tests soient revenus normaux, le cardiologue m'a suggéré de
faire un test de table basculante. C'est alors que j'ai reçu le diagnostic de
POTS. J'ai également consulté un rhumatologue et un neurologue, mais jusqu'à
présent, tous les tests ont été clairs. Seul mon ANA était à la limite du
positif.

Nous sommes maintenant en septembre, cinq mois plus tard, et je ressens toujours
des symptômes.

Malheureusement, de nombreux médecins ne savent pas comment me traiter ou
m'aider. Chaque professionnel consulté est différent, ils ne savent pas combien
de temps cela va durer, si cela va disparaître, si cela va s'aggraver ou devenir
quelque chose de permanent. Tout ce qu'ils disent, c'est “nous ne savons pas ce
que le vaccin Moderna peut faire, c'est trop récent pour le savoir”. C'est
vraiment un moment effrayant à vivre.

Cette expérience a complètement changé ma vie. Je n'ai pas pu travailler depuis
avril et je suis sur le point de perdre mon emploi. Chercher des médecins et des
spécialistes, prendre des rendez-vous, aller aux rendez-vous, faire le suivi des
médecins et faire des tests approfondis a été un travail à plein temps. Je suis
également une thérapie pour faire face à tout ce que je vis. Certains jours sont
“ bons ” et d'autres “ très mauvais ”.

Passer d'une vie active et en bonne santé à une vie de malade et de confinée à
la maison, sans aucune énergie, a été un véritable défi, difficile à relever. Ma
vie a littéralement changé du jour au lendemain.
