\testimony{Maddie de Garay}
          {Cincinnati, Ohio, USA}
          {12ans}
          {Pfizer}
          {janvier 2021}
          {troubles cardiaques, neurologiques, gastriques et urinaires, musculaires, épileptiques}
          {picture.jpg}
          {https://nomoresilence.world/pfizer-biontech/maddie-de-garay-pfizer-adverse-reaction/}
          {

Maddie a participé aux essais cliniques Pfizer menés auprès des jeunes de 12 à
15 ans par le biais du programme Gamble de l'hôpital pour enfants de Cincinnati.

Ses parents la décrivent comme une fille énergique et en bonne santé, une bonne
élève, gentille et sociable, mais ils n'ont pas vu cette version de Maddie
depuis qu'elle a reçu le deuxième vaccin Pfizer. En l'espace de 24 heures, elle
a développé des douleurs abdominales, musculaires et nerveuses qui sont devenues
iinsupportables et, au cours des deux mois et demi qui ont suivi, elle a été
admise à l'hôpital trois fois, chaque séjour étant un peu plus long que le
précédent.

Elle a développé des symptômes supplémentaires, notamment une gastroparésie, des
nausées et des vomissements, une pression sanguine et un rythme cardiaque
erratiques, des pertes de mémoire (elle confond les mots), un brouillard
cérébral, des maux de tête, des vertiges, des évanouissements (elle est tombée
et s'est cognée la tête) et des crises d'épilepsie, des tics verbaux et moteurs,
une perte de sensibilité à partir de la taille, une faiblesse musculaire, des
changements drastiques de sa vision, une rétention urinaire et perte de contrôle
de la vessie, des flux menstruels importants et très irréguliers et finalement
elle a dû se faire poser une sonde gastrique pour s'alimenter. Tous ces
symptômes sont encore présents aujourd'hui, et certains jours sont pires que
d'autres. Steph de Garay, la mère de Maddie, déclare : “Elle n'avait aucun
problème. Elle était parfaitement heureuse, et maintenant, quoi qu'il se soit
passé, cela l'a changée.

}
