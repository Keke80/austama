Apparemment, la myocardite est une réaction très rare du vaccin… et pourtant,
nous étions trois à l'hopital à avoir une myocardite inexpliquée. Et nous avions
tous les trois reçu une seconde dose de Pfizer récemment.

D'accord, je fais partie des malchanceux, mais à 36 ans, est-ce que les risques
du covid dépassent celui de développer une myocardite… certainement pas dans mon
cas.

J'ai pris les doses parce que je voulais partir à l'étranger, et à cause de
toute cette pression sur les passeports vaccinaux.

Mon sang a indiqué un taux élevé de troponin, et aussi des irrégularités sur mon
électrocardiogramme. J'ai donc fait une échocardiographie et une angiographie,
qui ont pu confirmer ma myocardite.

J'ai reçu ma dose il y a environ 8 semaines. Entre temps, j'ai eu des
palpitations mais je n'avais pas fait le rapprochement avec les doses. Jeudi
matin, je me suis réveillé avec une sérieuse douleur à la poitrine qui pulsait
tout le long de mon bras.

Moi même ayant 36 ans, Pawo 39 et Clive 41, nous avons été diagnostiqués avec
une myocardite, tous dans le même service, et aucun de nous n'avait de facteurs
d'infection.

