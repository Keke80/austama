Le gouvernement slovène suspend le vaccin Janssen Covid-19 après la mort de
Katja Jagodic, une étudiante de 20 ans.

Katja a été vaccinée suite à la mise en place de l'obligation pour les
universités en Slovénie. Elle a reçu son vaccin le jeudi 16 septembre 2021. Elle
a souffert de plusieurs problèmes de santé peu après et a été hospitalisée le
lundi 27 septembre au Centre clinique universitaire (UKC) de Ljubljana. Ses
symptômes étaient si graves que des soins médicaux immédiats ont été
nécessaires. Elle est décédée le 29 septembre au petit matin.

Igor Rigler, chef du service de neurologie d'urgence du centre clinique
universitaire, a déclaré : “Elle est tombée malade dans un état très grave. En
même temps, elle présentait des caillots sanguins et un faible taux de
plaquettes, ce qui a provoqué des saignements dans son cerveau. Nous avons fait
tout ce que la médecine moderne peut faire dans de telles conditions. Nous avons
consulté tous les experts impliqués dans ce domaine au centre hospitalier -
spécialistes vasculaires, hématologues, néphrologues et neurologues. Nous avons
abordé le traitement de manière agressive, mais malheureusement nous n'avons pas
pu l'aider.”

Le ministre slovène de la Santé Janez Poklukar a ordonné la suspension du vaccin
Janssen Covid-19 après la nouvelle.

Le père de la jeune femme de 20 ans décédée était présent lors d'une
manifestation le 29 septembre 2021 contre le “COVID-Pass”. Les participants au
rassemblement ont consacré une minute de silence avant de s'exprimer : “Elle
voulait avoir la liberté, comme tout le monde ici. Le 16 septembre, elle est
allée se faire vacciner chez Janssen pour ne pas être limitée par ces
bêtises. Elle n'est plus aujourd'hui. Elle avait 20 ans”.

Et il poursuit, “quand vous arrivez sur le lieu de vaccination, personne ne vous
prévient des complications possibles. Personne ne vous avertit que la
vaccination avec le vaccin Janssen n'est pas recommandée pour les personnes de
moins de 40 ans. Il n'y a que des chiffres. On ne parle que de pourcentages. Ma
Katja n'était pas un pourcentage. Elle était ma Katja”, a déclaré le père ému.

Après les discours, la foule s'est dirigée vers le palais présidentiel et a
poursuivi son chemin dans les rues de Ljubljana. Les manifestants, qui
s'opposent au COVID-Pass obligatoire pour travailler, accéder à une
station-service ou à un centre commercial, ont fini par bloquer une
autoroute. La police a été envoyée sur place avec des canons à eau pour
combattre les manifestants.

