\testimony{Lisa Stonehouse}
          {Age Non Communiqué}
          {Alberta, Canada}
          {AstraZeneca}
          {21 avril 2021}
          {thrombocytopénie thrombotique,
            décès (02 mai 2021)}
          {picture.jpg}
          {https://nomoresilence.world/astra-zeneca/lisa-stonehouse-died-from-astrazeneca-vaccine/}
          {

Wilf Lowenberg, un ami de longue date de la famille de Lisa, dit que Lisa
croyait avoir les mêmes symptômes que beaucoup d'autres avaient après la
vaccination, mais sa santé s'est dégradée. “Elle a appelé la ligne des urgences,
disant qu'elle souffrait de sévères maux de tête. Ils lui ont dit que ça
semblait être une réaction normale du vaccin, et que parfois ça pouvait mettre
jusque 14 jours pour s'en remettre”, raconte-t-il. On a dit à Lisa de prendre du
Tylenol, de se reposer et de boire de l'eau en attendant que la douleur se
dissipe. Mais elle n'est pas partie!

Pendant le diner le jour suivant, Jordan, la fille de 19 ans de Lisa, a dit à sa
mère qu'elle l'emmenait aux urgences puisqu'elle avait toujours de forts mau de
tête et qu'elle vomissait. En arrivant aux urgences, après juste cinq minutes,
ils l'ont renvoyé chez elle.

Lisa a dit à sa fille qu'elle avait raconté aux médecins ce qu'il se passait, et
ils ont répondu: “Non, nous ne pouvons pas vous aider. Ca me semble être des
symptômes tout à fait normaux. Prenez du Tylenol et, si ça ne s'aggrave pas,
alors dans une semaine ou deux revenez et voyez un docteur.”

Le lendemain, la fille de Lisa Stonehouse l'a emmené à un hôpital communautaire,
où un scanner a trouvé des caillots de sang dans son cerveau. Elle a fait une
crise d'épilepsie alors qu'elle a été transportée d'urgence vers l'hopital de
l'Université d'Alberta.

Lisa est restée sous assistance respiratoire pendant le week-end et a été
retirée de la machine. Docteur Deena Hinshaw a rapporté le lendemain la mort de
Lisa, disant que sa mort était dûe à un trouble de caillots sanguins appelé
Thrombocytopénie immunitaire prothrombotique induite par le vaccin - TIPIV (ou
TTIV).

}
