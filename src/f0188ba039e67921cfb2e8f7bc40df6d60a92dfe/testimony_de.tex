Der aus Blumenau , einer Gemeinde im südbrasilianischen Bundesstaat Santa
Catarina, stammende Rechtsanwalt Bruno Oscar Graf, 28, starb am Donnerstag, den
26. August, an den Folgen eines Schlaganfalls der durch eine Thrombose
verursacht wurde. Die Mutter des Jungen, Arlene Ferrari Graf, glaubt, dass der
Impfstoff von AstraZeneca, der gegen Covid-19 eingesetzt wird, den Tod ihres
Sohnes verursacht haben könnte. Er wurde am 14. August - zwölf Tage vor seinem
Tod - im Impfzentrum im Parque Vila Germânica geimpft.

Arlene sagt, dass Bruno starke Kopfschmerzen und Fieber hatte und deshalb am
Montag, dem 23. August, ins Krankenhaus eingeliefert wurde. Am Dienstag erlitt
er Berichten zufolge einen Schlaganfall mit mehreren Komplikationen und starb
zwei Tage später.

“Bei einer Blutuntersuchung wurden niedrige Thrombozyten und ein hoher CRP-Wert
festgestellt. Es bestand der Verdacht auf Covid oder Dengue. Sie gaben ihm nur
Medikamente gegen Kopfschmerzen. (…) Am Dienstag, um 19 Uhr, hatte er einen
Schlaganfall und es gab nichts mehr zu tun”, so die Mutter.

Über die sozialen Netzwerke veröffentlichte Arlene eine Todesnachricht:

“Bruno war, in jeder Hinsicht, ein perfekter Sohn. Er war ein liebevolles,
rücksichtsvolles, hilfsbereites und freundliches Kind. Gott in seiner
unendlichen Barmherzigkeit hat uns mit diesen fast 29 Jahren des glücklichen und
von Liebe bestimmten Zusammenlebens, erleuchtet. Es verging kein Tag, an dem ich
ihn nicht küsste und sagte: “Sohn, Mutter liebt dich. - Ich liebe dich auch,
Mama”… Ich liebe meine Kinder und Zuneigung und Küsse sind Teil unseres
täglichen Lebens”, schrieb sie in dem Beitrag.

In einem anderen Kommentar weist sie auch darauf hin, dass sie und Brunos Vater
beschlossen haben, seine Organe zu spenden, damit “andere Eltern lächeln können
und Brunos Herz weiterschlägt”.

Arlene gibt an, dass auf dem Totenschein steht, dass der Schlaganfall auf eine
immunthrombotische Thrombozytopenie (ITT) zurückzuführen ist. Bei ihren
Nachforschungen fand sie heraus, dass die Ursache für diese Thrombose der
Impfstoff sein könnte.

Außerdem führte sie an, dass bei Gesprächen mit den Ärzten, die Bruno
behandelten, einige bestätigten, dass es einen Zusammenhang mit dem Impfstoff
gab, und andere aus Angst nur nickten, als sie Fragen stellte.

Sie berichtete auch, dass eine Untersuchung zur Klärung der Frage, ob der
Impfstoff die Ursache der Thrombose war, bereits durchgeführt wurde und dass man
auf das Ergebnis wartet.

“Ich warte auf das Ergebnis eines Tests namens Anti-Heparin PF4, der nach Europa
ging, und das uns aufklären könnte. Aber ich stütze mich auf die alle Faktoren,
die zum Tod meines geliebten Sohnes geführt haben”, sagte sie.

Über eine Pressestelle erklärte das Krankenhaus Santa Catarina, in dem Bruno
verstorben war, dass es sich zum jetzigen Zeitpunkt nicht äußern wird, da es
noch auf die Ergebnisse der Untersuchungstests wartet.

Bruno Graf war gezwungen, sich impfen zu lassen, da er nach Europa ziehen
wollte, wo er heiraten und eine Familie gründen wollte.

Wie üblich wurde der Fall von den Mainstream-Medien ignoriert, die Regierung
teilte mit, dass der Nutzen die Risiken überwiege und es zu früh sei, um
Schlussfolgerungen zu ziehen. Die Untersuchung, die beweisen könnte, dass der
Tod durch den Impfstoff verursacht wurde, wurde von der Familie bezahlt und
kostete etwa 800 Dollar (mehr als drei Mindestlöhne in Brasilien)!
