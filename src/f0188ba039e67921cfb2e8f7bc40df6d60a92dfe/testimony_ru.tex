Адвокат и житель Блуменау (который находится в муниципалитете штата
Санта-Катарина) Бруно Оскар Граф скончался в четверг 26 августа в возрасте 28
лет в результате инсульта вызванного тромбозом. Мать молодого человека, Арлен
Феррари Граф, считает, что её сына убила вакцина AstraZeneca. Ему сделали
прививку 14 августа, за 12 дней до смерти, в Центре вакцинации, расположенном в
Parque Vila Germanica.

Арлин говорит, что у Бруно была сильная головная боль и высокая температура,
поэтому, он был госпитализирован в понедельник, 23-го. Во вторник у него
случился инсульт, который привёл к нескольким осложнениям в его состоянии и он
умер через два дня.

«Ему сделали анализ крови, который показал низкое количество тромбоцитов и
высокий уровень С-реактивного белка. Они заподозрили коронавирус или лихорадку
Денге. Они не оказали никакой помощи, кроме как дали ему лекарство от головной
боли. (…) Во вторник, в 7 утра, у него случился инсульт, и что-то сделать было
уже поздно», - сказала его мать.

В социальных сетях Арлин опубликовала сообщение о смерти сына:

«Бруно был идеальным сыном во всех отношениях. Любящий, заботливый, отзывчивый и
добрый ребенок. Господь, по безграничной своей милости, благословил нас этими 29
годами счастливого сосуществования, где безраздельно царила любовь. Не проходило
и дня, чтобы я не целовала его и не говорила: «Сынок, твоя мать любит тебя. - Я
тоже люблю тебя, мама.»… Я люблю своих детей, и сердечные отношения являются
частью нашей повседневной жизни», - написала она в своем аккаунте.

В другом посте она говорит, что отец Бруно решил пожертвовать органы его сына,
чтобы «другие родители могли радоваться, а сердце Бруно продолжать биться».

Арлин прокомментировала, что в свидетельстве о смерти указано, что инсульт был
вызван аутоиммунной тромбоцитопенией. Проведя своё расследование, она узнала,
что причиной этого тромбоза могла быть вакцина.

К тому же, она подняла эту тему с врачами, которые лечили Бруно, и некоторые
подтвердили, что связь с вакциной здесь действительно существует, а другие, из
страха, просто кивали, когда она задавала им этот вопрос.

Арлин также сообщила, что уже начато расследование связи вакцины и тромбоза, и
сейчас они ждут результатов.

«Я жду результата теста под названием анти-гепарин PF4, который был отправлен в
Европу, он должен нам помочь в расследовании. Но я принимаю во внимание все
факторы, которые привели к смерти моего любимого сына», - сказала она.

Госпиталь Санта-Катарина, место смерти Бруно, не захотел комментировать дело в
прессе, потому что они всё еще ждут результатов расследования.

Бруно Граф был вынужден пройти вакцинацию, поскольку ему предстояло переехать в
Европу, где он собирался жениться и создать семью.

Как обычно, основные средства массовой информации проигнорировали это дело, а
правительство сообщило, что преимущества вакцинации перевешивают риски и что
сейчас ещё слишком рано делать выводы. Тест, доказывающий, что смерть была
вызвана вакциной, был оплачен семьей и стоил ей около 800 долларов, что
равняется сумме трех минимальных заработных плат в Бразилии.


