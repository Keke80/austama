Rapporté par Marie, sa mère :

En quelques semaines, elle a commencé à bégayer et à avoir des mouvements de
tête incontrôlables.

“Elle ressemblait à quelqu'un qui a la maladie de Parkinson. Elle n'avait jamais
bégayé ou eu ces tics auparavant”.

Elle a été admise à l'hôpital où elle a passé deux nuits et subi de nombreux
tests, avant de sortir de l'hôpital et de se voir dire qu'il s'agissait d'un
“tic nerveux”, et de consulter un prestataire de santé mentale.

“Nous avons demandé plusieurs fois si cela pouvait être le vaccin et on nous a
tout simplement ignorés, jusqu'à ce qu'un médecin nous dise qu'il n'avait aucune
idée de ce que c'était, mais que ce n'était “absolument pas le vaccin” et que
nous ne pouvions pas tout mettre sur le compte de cela.”

Ses parents ont plaidé pour être orientés vers un neurologue, qui a diagnostiqué
chez Sarah un trouble fonctionnel du mouvement, et lui a dit que c'était “lié au
vaccin, mais pas que.” Ils ont également déclaré qu'il s'agissait d'un effet
secondaire “extrêmement rare”, bien qu'ils en aient vu plusieurs cas dans leur
propre cabinet au cours de l'année écoulée.

Sarah avait terminé l'année scolaire précédente avec une moyenne de 4,7 et était
inscrite dans un programme d'Early College, en bonne voie pour obtenir un
diplôme d'associé. Compte tenu de son état physique actuel et de ses
limitations, elle n'a eu d'autre choix que d'abandonner ses cours pour le
semestre à venir. Elle a commencé ses cours normaux, mais il lui est impossible
de baisser les yeux ou d'écrire sans déclencher de violents tremblements et
spasmes. Son professeur va taper ses notes pour elle.

“J'ai le cœur brisé parce qu'elle a travaillé si dur et que tout a changé pour
elle - et je suis tellement en colère ! Nos vies entières ont changé, et pour
quoi ? Un vaccin qui ne fonctionne même pas ! J'espère que vous, le lecteur,
serez en mesure de prendre une décision éclairée lorsque vous déciderez de vous
faire vacciner ou non. Nous n'avons pas eu cette possibilité.”


