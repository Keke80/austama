Mon nom est Jordan Vasquez, j'ai 26 ans, je vis actuellement à San Francisco en
Californie et je suis étudiant. Je fais aussi des compétitions d'arts
martiaux. Donc, j'ai pris la première dose du vaccin pfizer le 2 juillet et j'ai
eu des sortes de sensation, des effets le jours suivant. Le 3 juillet quand je
me suis levé le samedi matin, ma poitrine était fermée et nouée et il se passait
quelque chose de bizarre dans la partie gauche de ma poitrine. Deux jours après
cela a commencé à se serrer encore plus, et je me suis dit “huu peut-être que je
devrais aller faire vérifier ça”. Donc, je suis partis faire un encéphalogramme
et trois autres tests, des ultrasons et ensuite des rayons X, puis ils m'ont
diagnostiqué une péricardite.

Ils m'ont dit qu'ils voulaient d'abord en être sûrs. Ils voulaient savoir si
c'était dû à l'exercice ou aux arts martiaux mixtes. À ce moment-là, je leur ai
dit que c'était surtout l'injection parce que, vous savez, nous nous sommes
entraînés toute l'année avec des centaines de personnes. Les arts martiaux
mixtes sont un sport dans lequel, vous savez, vous ne pouvez pas porter de
masque lorsque vous luttez, vous vous battez. Donc, j'ai fait toute l'année, je
n'ai pas eu de problème, surtout avec quelque chose au niveau de ma poitrine.

Je suis définitivement pro-vax pour sûr, j'ai pris tous les vaccins et, je
voulais juste le dire. Pour celui-ci, je me suis inquiété de la rapidité avec
laquelle il a été fabriqué. Il n'a pas été approuvé par la FDA. Je pense qu'ils
voulaient plus d'études dessus […] faire plus de recherches avant de vacciner
toute la population. Je pense que la transparence est une chose importante à
notre époque. Définitivement. Plus de tests, plus de recherches, c'est en
quelque sorte les bases de la science et comment la science fonctionne. La
recherche, les études et le fait de mettre différentes personnes dans
différentes situations ou environnements. Ce sont les bases de la science.

