Ma mère est tombée malade le 16 juin suite à la deuxième dose du vaccin Pfizer
qu'elle avait reçue le 9 juin. Au début, nous avons pensé qu'elle avait juste un
petit coup de mou, mais ça s'est rapidement aggravé, au point qu'elle ne pouvait
plus marcher et souffrait énormément. Elle pleurait constamment, ne pouvait pas
se nourrir seule, ne pouvait pas aller aux toilettes toute seule et nous devions
l'aider tout le temps.

Le 18 juin, nous étions en train de nourrir ma mère lorsque j'ai décidé de
prendre sa température. Sa température était de 102,9 (fahrenheit, environ 39,5
en degrés celsius). C'est alors que nous avons décidé d'appeler les
urgences. Une fois à l'hôpital, ils l'ont mise en quarantaine et lui ont fait
passer un test Covid, mais ils n'ont pas pu trouver ce qui n'allait pas. Il leur
a fallu littéralement une semaine ou deux pour établir un diagnostic.

Cette injection lui a donné une maladie appelée polyneuropathie, c'est-à-dire
une douleur chronique. Faiblesse neuromusculaire, hypertension, hyperlipidémie,
neuropathie périphérique, débilité fonctionnelle, diabète de type 2
insulinodépendant et faiblesse généralisée. Je sais que c'est beaucoup à
comprendre, donc en gros elle ne pouvait pas marcher, elle ne pouvait pas lever
les bras, elle ne pouvait pas attraper quoi que ce soit, elle ne pouvait pas se
nourrir ou remuer les orteils. J'ai littéralement vu ma mère passer d'une
personne active, en bonne santé et aimant s'amuser à un “légume”, allongée sur
un lit d'hôpital, pleurant dans une agonie extrême.

Etant son enfant, je ne pouvais pas y croire. J'ai cru que j'allais perdre ma
mère et je ne savais pas quoi faire. Nous avons essayé de contacter les chaînes
d'informations, mais ils ont refusé de publier l'histoire parce que le président
met tellement l'accent sur la vaccination et ses bienfaits qu'ils ne veulent pas
publier de mauvaises histoires.

Nous avons également contacté Pfizer, qui lui a fait la piqûre, mais ils ne nous
aident pas non plus.

