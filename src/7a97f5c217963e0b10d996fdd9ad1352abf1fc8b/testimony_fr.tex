Shannon raconte son histoire :

Lorsque je me suis assise et que j'ai pensé au vaccin, j'étais nerveuse et je
n'avais pas vraiment envie de le recevoir. Après avoir vu que tout le monde
l'avait reçu et s'en était sorti, j'ai décidé que j'étais prête. Le 21 décembre
2020, j'ai reçu ma première dose de Pfizer et ma seconde dose le 11 janvier
2021.

J'ai envoyé un message à mon mari pour lui dire que je ne me sentais pas bien et
qu'il était mon contact d'urgence au cas où quelque chose arriverait. J'ai donc
marché jusqu'à l'unité de soins intensifs où mes collègues étaient assis. J'ai
demandé à l'un d'eux s'il pouvait prendre ma tension artérielle, ce qu'elle a
fait. C'était un peu trop faible.

Ensuite, je me suis retrouvée aux urgences avec des infirmières venues de
partout. On m'a dit que j'avais des crises et qu'il fallait me transférer dans
un hôpital plus grand. J'ai fais environ 25 crises au total pendant les trois
heures où je suis restée aux urgences.

Ils m'ont fait sortir assez rapidement et je suis repartie en ambulance. J'ai eu
trois crises en l'espace de 10 minutes environ. L'ambulancier a appelé le grand
hôpital où je me rendais et on lui a dit de m'intuber. En tant que thérapeute
respiratoire, je savais que ce n'était pas une très bonne idée.

L'ambulancier essaya autre chose et si cela ne fonctionnait pas, il
m'intuberait. Le tranquillisant que l'ambulancier m'a donné a fonctionné (Dieu
merci). J'ai été envoyé à l'unité de soins intensifs du grand hôpital et après
deux jours et demi d'hospitalisation, un EEG normal avec activité épileptique et
une IRM normale, j'ai été renvoyé chez moi.

Après mon retour à la maison, je suis passée du statut de jeune femme de 22 ans
indépendante et en bonne santé à celui de personne très dépendante des soins de
mon mari. Jamais je n'aurais pensé qu'une chaise de douche devrait être utilisée
à mon âge. Mon mari devait m'aider à me laver les cheveux à cause de ma
faiblesse, il devait me retenir physiquement pour que je ne me tombe pas du lit
à 3 heures du matin, ou il devait m'attraper le bras et m'aider à marcher pour
que je ne m'écroule pas.

Après avoir visité le plus grand hôpital, je suis allée dans deux autres encore
plus grands. Tout le monde me donnait toujours les mêmes réponses : cause
inconnue, cause inconnue, et encore une fois cause inconnue. Après des mois et
des mois de douleur et de souffrance, j'ai poussé une gueulante et, dans
l'espoir d'une réponse, je suis allé voir un autre médecin. Ce médecin a fait un
examen neurologique et a déterminé que la cause de mes crises était une
inflammation du cerveau causée par une réaction au vaccin COVID-19.

Si vous entrez dans chez nous en ce moment, vous verrez que toutes les tables
sont contre le mur. J'essaie de rester dans des endroits sûrs de notre maison
pendant que mon mari est parti travailler pour gagner notre vie. À 22 ans : Je
ne peux pas occuper un emploi à temps plein, je ne peux pas conduire, la plupart
des jours je ne peux pas faire une journée complète d'activité, parfois, il me
faut utiliser toute mon énergie pour préparer le dîner.

Certains jours, je ne fais que deux crises et d'autres jours, je ne peux pas les
compter parce qu'elles sont continues.

Les sept derniers mois ont été absolument terribles à vivre, mon corps me
faisait mal d'une manière que je n'aurais jamais pu imaginer. Mon médecin dit
que c'est quelque chose avec lequel je devrai probablement vivre le reste de ma
vie.

J'ai gardé le silence assez longtemps par peur des commentaires. Je ne peux pas
parler au nom de tous ceux qui ont eu des réactions, mais ce n'est que mon
histoire. Si vous faites l'expérience de tels effets, vous n'êtes pas seul.

