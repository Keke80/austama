Je m'appelle Chantal. Je suis un officier de police de 37 ans, originaire
d'Australie occidentale, et j'ai eu une réaction grave après avoir reçu le
vaccin Pfizer.

En août, mon employeur a annoncé que toute personne qui n'était pas vaccinée
contre le COVID serait traitée différemment en devant porter des masques en
permanence sur le lieu de travail et en étant exclue des bâtiments. Les
personnes qui n'ont pas été vaccinées se verront également retirer des postes
pour lesquels elles ont travaillé dur pour les remplacer par des postes de
bureau (tout cela a été confirmé par un membre du personnel des relations
industrielles). Cette annonce n'est pas un secret pour la communauté puisqu'elle
a été publiée en première page du journal West Australian.

Avant cette annonce, j'avais décidé de ne pas me faire vacciner. J'ai le droit
de faire mon choix. Veuillez noter que je ne suis pas une anti-vaxx comme
certains le disent, mais que je prends mes décisions en fonction des
informations dont je dispose et que je croyais et que crois toujours ne pas
avoir eu.

Dix minutes après avoir reçu le vaccin Pfizer, j'ai eu des vertiges et des
nausées, 15 minutes plus tard, j'avais de l'urticaire partout sur moi. Le
médecin et l'ambulancier ont maîtrisé l'éruption cutanée, mais dès mon retour à
la maison, j'ai eu de la fièvre, des frissons et je me suis sentie très
mal. Pendant les trois semaines et demie qui ont suivi, j'ai souffert
d'éruptions cutanées tous les jours, de fièvres allant jusqu'à 39,7, de douleurs
musculaires, de symptômes semblables à ceux de la grippe, de nausées, de
vomissements, de diarrhée, de pression dans les sinus et d'une toux si forte que
j'avais l'impression que mes vaisseaux sanguins allaient m'exploser au
visage. Certains jours, je pleurais parce que je pensais que j'allais
mourir. J'étais tellement malade. Lorsque j'ai cru que je commençais à aller
mieux, mais que j'étais toujours extrêmement fatiguée et que j'avais constamment
la tête qui tournait, je suis retournée au travail, mais j'ai passé beaucoup de
temps allongée sur le sol de mon bureau à lutter pour passer la journée.

Un mardi soir, j'ai remarqué que mes yeux étaient bizarres. Ils étaient lourds
et j'avais l'impression qu'il fallait les frotter tout le temps. Le lendemain
matin, je me suis réveillée pour aller travailler et un œil ne s'ouvrait pas et
l'autre ne se fermait pas. Celui qui ne se fermait pas ne clignait pas non
plus. Je pensais que j'avais des allergies, j'ai donc pris un antihistaminique
et j'ai conduit une heure pour aller travailler. À 9 heures, j'ai senti ma lèvre
supérieure devenir un peu bizarre et ma collègue de travail m'a dit que mon
visage s'affaissait sur le côté et qu'elle m'emmenait aux urgences. Je lui serai
toujours reconnaissante de m'y avoir emmenée, car elle m'a sauvé la vie. Une
minute après mon arrivée aux urgences, tout le monde se précipitait autour de
moi et je commençais à paniquer. La seconde d'après, on m'a branché à des
machines et un spécialiste des accidents vasculaires cérébraux ainsi que de
nombreux autres médecins et infirmières étaient dans mon box. J'ai été admis à
l'hôpital.

Le lendemain matin, je me suis levée et j'ai pris une douche, mais j'ai fini par
avoir une attaque. J'étais seule dans la douche et mon corps a commencé à
s'agiter de façon incontrôlable et le côté droit de mon corps a eu une sensation
bizarre. Je ne pouvais plus me tenir debout et j'étais par terre, effrayée. Mon
côté droit est devenu très lourd et s'est engourdi avec une sensation bizarre de
picotement. À ce stade, le côté gauche de mon visage était complètement paralysé
et j'avais une faiblesse musculaire extrême dans mon bras et ma jambe gauche.

On m'a fait passer d'autres examens qui ont montré que l'artère principale de
mon cerveau s'était rompue et que j'avais subi un mini-AVC (AIT). Ce jour-là,
j'ai été transférée dans le service des accidents vasculaires cérébraux d'un
autre hôpital et, à partir de ce moment-là, j'ai été suivie toutes les heures.

Je ne peux pas vous dire à quel point c'était effrayant d'être dans un service
d'accident vasculaire cérébral avec tous les patients âgés, en me demandant ce
que je faisais ici. On m'a fait passer des tests de dépistage de toutes les
maladies et de toutes les carences nutritionnelles susceptibles de provoquer un
accident vasculaire cérébral, et le médecin a déclaré que j'étais en parfaite
santé et qu'il n'y avait aucune cause possible à ce qui m'était arrivé.

Pendant mon séjour à l'hôpital, j'étais surveillée toutes les heures. C'était
fou. Je ne pouvais pas dormir et j'avais constamment peur que cela se
reproduise. Je ne pouvais ni manger ni boire correctement et je bavais tout sur
le côté de mon visage. C'était humiliant. Pour les médecins, ce n'était qu'un
problème mineur, mais pour une jeune femme de 37 ans, c'était très grave. On m'a
dit qu'il y avait 25\% de chances qu'elle ne guérisse pas et que si elle
guérissait, cela prendrait des mois. En soi, c'était extrêmement pénible.

Je pleurais plusieurs fois par jour. C'était mentalement difficile de supporter
chaque moment. Au début, je n'avais pas le droit de bouger, mais les infirmières
m'ont permis d'aller aux toilettes parce qu'aller aux toilettes dans un plateau
était trop perturbant pour moi. Si je bougeais trop ou trop vite, je pouvais
avoir une autre attaque. Je n'ai pas pu recevoir de traitement car les médecins
ont jugé que c'était trop dangereux. La seule chose que les médecins et moi
pouvions faire était d'attendre et d'espérer que mon artère se guérisse
d'elle-même. Cela prendra beaucoup de temps et, en attendant, mes activités
habituelles sont suspendues.

Lorsque je suis sorti de l'hôpital, j'étais vraiment heureuse car je ne
supportais plus d'être enfermé à l'hôpital. Ne vous méprenez pas, mes
infirmières et mes médecins étaient extraordinaires, je ne peux rien dire de mal
sur leurs efforts et leur nature attentionnée. Cependant, c'était tellement
effrayant de se retrouver à la maison sans soins constants. Je n'avais pas le
droit de rester seule et le risque de subir un autre AVC était et est toujours
très élevé. J'étais constamment dans la peur et même si j'ai de plus en plus
confiance en moi, je m'inquiète toujours à chaque instant. Je suis de plus en
plus éloignée des soins immédiats et le risque de lésions cérébrales permanentes
est très réel. Je souffre également de saignements de nez constants et
d'articulations douloureuses.

J'ai pris rendez-vous à la clinique de sécurité vaccinale. Malheureusement, il y
avait une vingtaine d'autres femmes du même âge que moi qui attendaient. Je n'ai
pas vraiment compris l'objet du rendez-vous jusqu'à ce qu'on m'emmène dans une
pièce privée avec un médecin qui a essayé de me dire que le vaccin Pfizer
n'avait rien à voir avec ce qui m'était arrivé, mais qui ne pouvait pas non plus
me dire que ce n'était pas le cas. Il s'est assis sur sa chaise et a déclaré
qu'il valait la peine de courir le risque d'avoir un autre AVC pour recevoir ma
deuxième injection Pfizer. Comment un médecin peut-il s'asseoir là et dire de
telles choses à quelqu'un ? Il était prêt à risquer ma vie pour atteindre ce que
je crois être son objectif, à savoir faire vacciner le plus de gens possible. Il
ne s'est pas soucié de ma sécurité et a même demandé, après que j'ai refusé le
vaccin, si je voulais qu'il me rappelle dans trois mois pour voir si j'avais
changé d'avis. Il n'a eu aucun respect pour ma décision.

L'aspect mental de cette situation est très difficile et m'a demandé beaucoup de
force. Je suis une personne très active et occupée et passer de cela à ne
pouvoir que marcher dans la maison est très pénible. Je dois faire très
attention à ma santé mentale et je ne remercierai jamais assez mon partenaire et
mes amis de tout laisser tomber pour m'aider et me soutenir. J'ai beaucoup de
chance.

Je ne veux rien tirer de mon histoire, à part la reconnaissance du fait qu'aucun
vaccin ou procédure médicale n'est sûr pour tout le monde. Ce n'est pas mon
opinion, c'est un fait, et le vaccin COVID ne fait pas exception. Personne n'a
le droit de dire à quelqu'un d'autre qu'il doit mettre quelque chose dans son
corps, car il ne connaît pas les risques pour cette personne. Cela provoque une
triste division dans notre société et ne rend personne heureux. Si vous
choisissez de vous faire vacciner, c'est très bien, et si vous ne le faites pas,
c'est aussi bien. Soyez gentils les uns envers les autres et traitez-les de
manière juste et équitable, nous le méritons tous.

Sur la photo, je suis à l'hôpital et je présente une paralysie faciale du côté
gauche. Malgré tout ce que les médecins m'ont dit, je suis de retour à l'hôpital
avec une infection de la vésicule biliaire. Ils doivent m'enlever la vésicule
biliaire mais ne peuvent pas le faire pour le moment à cause de tout ce qui se
passe. J'ai eu des douleurs extrêmes et j'ai vomi du sang. Je suis sous
antibiotiques en IV toutes les 8 heures.

À toutes les personnes qui m'ont dit que mon expérience n'était pas réelle et
que cela ne pourrait jamais être vrai. Avant de dire à quelqu'un qu'il est un
menteur ou d'exprimer votre opinion, demandez-vous deux choses : premièrement,
est-ce que j'ai toutes les informations nécessaires pour exprimer mon opinion et
contester l'expérience d'une autre personne et deuxièmement, est-ce que je suis
qualifié pour exprimer cette opinion.
