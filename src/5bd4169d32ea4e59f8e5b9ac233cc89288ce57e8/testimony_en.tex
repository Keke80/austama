Melle Stewart was a healthy, talented 40-year-old Australian actor living and
working in London with her husband, actor Ben Lewis. On May 24th 2021 Melle
received her first dose of the AstraZeneca Covid-19 Vaccine. Two weeks later on
June 8th, Melle’s life changed.

Fourteen days after receiving her first dose of the AstraZeneca vaccine, Melle
woke up at midnight to a strange feeling on the right-hand side of her body. She
tried to get out of bed but quickly collapsed. Melle was rushed to hospital
where her condition quickly deteriorated, losing all movement on the right-hand
side of her body as well as her ability to speak. She began having seizures and
was taken by ambulance to St George’s Hospital, London, where Neurosurgeons
battled to save her life, removing a large part of her skull to reduce the
pressure in her brain.

After being admitted to the neurological ICU, Melle was formally diagnosed with
Vaccine-Induced Thrombocytopenic Thrombosis (VITT), a condition that is now
acknowledged by the manufacturer AstraZeneca. Unfortunately, Melle had had a
severe stroke, caused by two large clots in the main veins of her brain, with
the pressure and low platelets causing a secondary bleed in the left frontal
lobe. Melle spent 3 weeks in an induced coma during which time she was
ventilated and received a number of blood transfusion procedures to help remove
the antibodies created by the vaccine that were causing the clots. She was also
put on heavy doses of anti-clotting and anti-seizure medication, which continue
to this day.

After 4 and a half weeks in the ICU, Melle’s fighting spirit that we all know
and love saw her finally released from the ICU and transferred to an Acute
Stroke Unit to commence her rehabilitation. On September 8th, exactly three
months after being admitted to hospital, Melle was then transferred to a
specialised rehabilitation hospital in London where she will undergo more
intensive treatment.

The devastation caused by VITT has been profound. Prior to vaccination, Melle
was a fit, healthy 40-year-old, a successful professional actress, dancer and
singer who had never been in hospital before. Due to the seriousness of Melle’s
condition, she will remain in hospital into 2022. She continues to work hard
every day relearning how to walk, talk and move her right arm and hand. Melle, a
vibrant storyteller with an undying passion for language and expression, is now
re-learning how to make sounds, form words and speak again. Her road to recovery
will be a long one including further surgery to fit a titanium plate to replace
the portion of her skull previously removed during surgery.

