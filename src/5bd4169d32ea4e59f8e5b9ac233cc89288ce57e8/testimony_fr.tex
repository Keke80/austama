\testimony{Melle Stewart}
{Londres, Royaume-Uni}
{40 ans}
{AstraZeneca}
{24 mai 2021}
{Thrombocytopénie induite par le vaccin, opération du cerveau}
{picture.jpg}
{https://nomoresilence.world/astra-zeneca/melle-stewart-astrazeneca-severe-adverse-reaction/}
{

Melle Stewart était une actrice australienne de 40 ans, talentueuse et en bonne
santé, qui vivait et travaillait à Londres avec son mari, l'acteur Ben Lewis. Le
24 mai 2021, Melle a reçu sa première dose du vaccin Covid-19
d'AstraZeneca. Deux semaines plus tard, le 8 juin, la vie de Melle a changé.

Quatorze jours après avoir reçu sa première dose du vaccin AstraZeneca, Melle
s'est réveillée à minuit avec une sensation étrange sur le côté droit de son
corps. Elle a essayé de sortir du lit mais s'est rapidement effondrée. Melle a
été transportée d'urgence à l'hôpital où son état s'est rapidement détérioré,
perdant l'usage du côté droit de son corps ainsi que sa capacité à parler. Elle
a commencé à avoir des crises et a été transportée en ambulance à l'hôpital St
George de Londres, où des neurochirurgiens se sont battus pour lui sauver la
vie, en retirant une grande partie de son crâne pour réduire la pression dans
son cerveau.

Après avoir été admise à l'unité de soins intensifs neurologiques, Melle a reçu
un diagnostic formel de thrombocytopénie induite par le vaccin (TITV), une
affection désormais reconnue par le fabricant AstraZeneca. Malheureusement,
Melle avait subi un grave accident vasculaire cérébral (AVC), causé par deux
gros caillots dans les veines principales de son cerveau, la pression et le
faible taux de plaquettes ayant provoqué une hémorragie secondaire dans le lobe
frontal gauche. Melle a passé trois semaines dans un coma artificiel pendant
lesquelles elle a été sous respiration artificielle et a reçu plusieurs
transfusions sanguines afin d'éliminer les anticorps créés par le vaccin qui
provoquaient les caillots. On lui a également administré de fortes doses
d'anticoagulants et d'anticonvulsivants, qui lui sont encore administrés
aujourd'hui.

Après quatre semaines et demie aux soins intensifs, la combativité de Melle, que
nous connaissons et aimons tous, lui a permis de sortir de l'unité de soins
intensifs et d'être transférée dans une unité de soins actifs pour les accidents
vasculaires cérébraux afin de commencer sa rééducation. Le 8 septembre,
exactement trois mois après son admission à l'hôpital, Melle a été transférée
dans un hôpital de rééducation spécialisé à Londres, où elle suivra un
traitement plus intensif.

La dévastation causée par le VITT a été profonde. Avant la vaccination, Melle
était une femme de 40 ans, en bonne santé et en pleine forme, une actrice,
danseuse et chanteuse professionnelle à succès qui n'avait jamais été
hospitalisée auparavant. En raison de la gravité de son état, Melle restera à
l'hôpital jusqu'en 2022. Elle continue à travailler dur chaque jour pour
réapprendre à marcher, à parler et à bouger son bras et sa main droits. Melle,
une conteuse pleine de vie avec une passion éternelle pour le langage et
l'expression, est en train de réapprendre à faire des sons, à former des mots et
à parler à nouveau. Son chemin vers la guérison sera long et elle devra subir
d'autres interventions chirurgicales dont une installer une plaque de titane
afin de remplacer la partie du crâne enlevée lors de l'opération.

}
