Je suis diabétique de type 1, mais j'ai reçu ma première injection du vaccin
Pfizer le 1er octobre 2021. Jusqu'à cette date, j'étais en bonne santé et tout
allait bien.

Le 17 octobre, je suis tombée très malade : frissons, fièvre et vomissements,

Le 18 octobre, je suis allé aux urgences et on m'a diagnostiqué une infection
bactérienne à E.coli et une infection rénale (je n'ai aucune idée de l'origine
de tout cela), j'ai donc été admis à l'hôpital.

Là-bas, j'avais tellement mal et plus rien ne fonctionnait, qu'on m'a donné un
autre médicament contre la douleur auquel je suis apparemment allergique. J'ai
failli mourir, allongée là, engourdie, avec cinq infirmières debout devant moi,
sans pouvoir bouger ni parler pendant une heure. Après mon rétablissement, on
m'a diagnostiqué une pneumonie et j'ai développé un bégaiement... sans oublier
que l'hôpital m'a injecté 7 litres de liquide, de sorte que tout mon corps a
enflé de la tête aux pieds.

Deux semaines et demie plus tard, j'ai pris quatre antibiotiques différents,
dont aucun ne semble fonctionner... Je souffre toujours beaucoup de la pneumonie
et du bégaiement.

J'ai demandé d'autres tests, mais tous les résultats sont normaux, à l'exception
d'une infection rénale et je crois que j'ai toujours une infection bactérienne à
E. coli. Le bégaiement ne s'est pas amélioré non plus. J'ai essayé de leur dire
que je pense que c'est le vaccin Pfizer, car j'étais en parfaite santé avant
cela, mais ils ne font rien pour moi !

Cela fait presque trois semaines et j'ai encore de fortes douleurs, une migraine
qui ne veut pas disparaître et je ne peux pas parler correctement.
