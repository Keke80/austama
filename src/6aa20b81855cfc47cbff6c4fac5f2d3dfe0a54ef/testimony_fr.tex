Brittany a fait neuf crises d'épilepsies en 24 heures, d'un grave traumatisme
neurologique et a été hospitalisée cinq fois en quatre semaines - tout cela
depuis qu'elle a reçu le vaccin Pfizer.

Brittany a reçu sa première injection Pfizer le lundi 9 août, dans un Walgreens
local. En rentrant chez elle en voiture, elle a été prise d'étourdissements et
d'un violent mal de tête. Elle a également ressenti des engourdissements et des
picotements dans ses extrémités, mais a considéré qu'il s'agissait d'effets
secondaires normaux après les injections. Nous nous sommes donc endormis en
pensant qu'elle se réveillerait dans un état normal. Ce n'était pas le cas.

Elle a souffert d'un engourdissement de tout le corps en plus de violents maux
de tête, de sensations de picotements dans les membres, d'une douleur aiguë
derrière les yeux et de confusion le lendemain matin. Elle s'est rendue aux
urgences, a été testée pour le COVID-19 et renvoyée chez elle. Une infirmière
lui a dit : “il y a trop de patients positifs au COVID ici en ce moment, nous ne
voulons pas vous exposer potentiellement. Revenez si les symptômes s'aggravent.”

Elle s'est allongée pour faire une sieste en rentrant chez elle et s'est
réveillée deux heures plus tard pour aller aux toilettes. L'instant d'après,
elle s'est réveillée à l'hôpital. Son petit ami l'a trouvée dans la salle de
bains, gisant sur le sol, dans une mare d'urine alors qu'elle perdait le
contrôle de sa vessie. Les ambulanciers l'ont ramenée au St. Lucie Medical
Center, le même endroit qui l'avait renvoyée chez elle quelques heures plus tôt.

Incapable de parler, les médecins lui ont fait passer un scanner cette
fois. Mais ils ont dit que tout était normal alors que son état était tout sauf
normal. Elle n'avait aucune idée de ce qui lui était arrivé ni de la façon dont
elle était arrivée à l'hôpital. Les infirmières ne cessent d'entrer dans la
chambre et de lui poser des questions. Mais elle ne pouvait pas parler, se
sentant prisonnière de son propre corps. Au bout d'une heure environ, elle est
finalement parvenue à cligner des yeux pour signaler oui et non et à utiliser un
bloc-notes et un stylo pour communiquer. Elle a écrit ce qui suit :

“Je sais ce que je veux dire, je peux vous entendre et vous comprendre. Mais je
n'arrive pas à faire sortir les mots. Je ne sais pas ce qui se passe”.

Les infirmières ont admis qu'elle pouvait avoir une réaction à l'injection
Pfizer. Mais elles lui ont tout de même “fortement recommandé” de rentrer chez
elle, car il y a beaucoup de patients sous COVID à l'hôpital. Le lendemain, le
11 août, après avoir compris que le St. Lucie Medical Center était une
fumisterie, Mme Jouppi s'est rendue au Cleveland Clinic Traditional
Hospital. Les médecins ont effectué une batterie de tests - IRM, radiographies,
électrocardiogrammes, échographies, etc. C'est l'électroencéphalogramme (EEG)
qui a révélé une activité épileptique. Les médecins ont prescrit des médicaments
anti-crises.

Brittany écrit :

“Ce n'est pas facile à partager, je me suis fait la promesse de sensibiliser les
gens et d'entrer en contact avec d'autres personnes qui ont des problèmes
neurologiques après avoir été vaccinées.

J'ai été admise aux urgences la nuit dernière, j'ai eu 9 crises dans les
dernières 24 heures et j'utilise actuellement un déambulateur pour me déplacer,
avec des jambes très tremblantes quand je suis debout. Je suis toujours en train
de comprendre ce qui se passe. Je suis sorti de l'hôpital et je vais suivre une
thérapie physique à domicile 3x par semaine et une étude des nerfs la semaine
prochaine.”
