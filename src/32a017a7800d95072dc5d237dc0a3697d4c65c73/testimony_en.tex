\testimony{Kendra Lippy}
{USA}
{Unknown}
{johnson \& johnson}
{March 2021}
{-}
{picture.jpg}
{https://nomoresilence.world/johnson/kendra-lippy-johnson-johnson-severe-adverse-reaction/}
{

Kendra was a healthy 38-year old woman - until she got the Johnson \& Johnson
(J\&J) COVID vaccine. Within about one week, she began experiencing headaches,
abdominal pain and nausea.

Kendra was diagnosed with severe blood clots that subsequently sent most of her
organs into failure. She also was left without most of her small intestine  and
with crippling medical bills that she said the federal government should
compensate her for.

Her blood clots developed in March. She was hospitalised for 33 days, including
22 days of intensive care.  Kendra now is in occupational and physical therapy,
and is working to regain basic functions, such as walking 20 minutes at a time
or climbing stairs. She had to relearn fine motor skills, including writing and
using a fork, and she had to relearn how to walk. She is reliant on total
parenteral nutrition (TPN), a feeding method that bypasses the gastrointestinal
tract.

I'm always going to have this disability ... that's going to limit what I can eat
and limit ... some activities that I can't do anymore, Kendra said. ``Right now, I
know it's hindering me being able to go back to work, which is what I want to
do. I'm not a stay-at-home person. I'm not somebody that's gonna sit still, it's
just not me. I have to do something.''  Part of her road to recovery includes
figuring out how to pay her extensive medical bills, which add up to more than
\$1 million.

Kendra wants to see a federal compensation system that is fair to her and others
who are harmed by COVID vaccines. Because the government shielded vaccine makers
from liability, she can't sue J\&J. She also doesn't have a legitimate legal
route to sue the government.

The CICP does not compensate anyone for non-economic damages, such as pain,
suffering and disfigurement, said Stephen Justino, Lippy's lawyer. ``It's a
woefully inadequate system,'' Justino said.

Attorney Rene Gentry, director of the Vaccine Injury Litigation Clinic at
George Washington University, said Congress must act to allow COVID vaccine
claims to go through a more transparent process. Gentry said that would be the
easiest way to ensure the system is fair for people like Kendra.

Gentry expects Congress will allow a more transparent process eventually, but it
may not help people like Kendra who have already been injured, she said.

According to Bowman, in order for the NVICP system to accept COVID vaccination
injury claims, the vaccines must be recommended for children and pregnant women
and must meet specific criteria.

}
