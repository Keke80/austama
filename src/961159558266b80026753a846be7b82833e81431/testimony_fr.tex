L'histoire d'Adriana racontée avec ses propres mots:

J'ai 32 ans et je suis mère célibataire d'un fils de 12 ans. J'ai récemment vécu
un terrible incident au cours duquel j'ai dû lutter pour ma vie après avoir reçu
ma première dose du vaccin Pfizer au travail.

J'ai dû subir d'importants tests et procédures, être intubée et ramenée à la
vie, passant un certain temps dans le coma puis aux soins intensifs. Depuis, je
suis sorti de l'hôpital et je me remets à la maison en suivant des séances de
physiothérapie et d'ergothérapie.

J'ai dû réapprendre à marcher, à parler, à écrire et à tout réapprendre. J'ai
également perdu la mémoire pendant un certain temps, mais je la retrouve chaque
jour.

Malheureusement, la réaction médicale au vaccin Pfizer m'a laissé avec de
lourdes factures médicales. Bien que je sois couvert par une assurance, pour une
raison quelconque, celle-ci ne couvre pas une grande partie des dépenses. Mes
factures médicales sont en suspens et dépassent maintenant 160000 dollars,
montant qui ne cesse d'augmenter. Je continue à appeler et à faire appel, mais
on me refuse toujours. J'ai en fait une assurance GHI pour être exact que je
reçois de mon travail au département correctionnel de New York, donc j'ai du mal
à comprendre pourquoi tout est refusé à gauche et à droite.

