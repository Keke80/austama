Шерил Коэн, здоровая 64-летняя женщина из Флориды, умерла через три месяца после
получения второй дозы вакцины против COVID от Pfizer.

Джанни Коэн сказала, что у её матери развилась болезнь Крейтцфельдта-Якоба (БКЯ)
— редкое, дегенеративное и смертельное заболевание головного мозга — вскоре
после вакцинации.

Джанни сказала, что её мать получила первую дозу Pfizer 5 апреля, а вторую — 25
апреля.

6 мая у Шерил случился первый эпизод, указывающий на то, что «что-то не так с
центральной нервной системой», — объяснила г-жа Джанни. «У неё был туман в
голове и полуобморочное состояние. Она не узнавала дорогу домой, куда ехала, и
была очень напугана».

31 мая Шерил позвонила в 911 из-за сильной головной боли. Её доставили в
медицинский центр North Shore в Хомстеде, Флорида, где она провела в больнице 10
дней.

Джанни сказала: «Она была доставлена в эту больницу, и я не знаю, что они нашли
у неё, но они продержали её 10 дней, а затем отправили домой. Она была в очень
плохом состоянии. Она сказала: «Эй, я не узнаю, где я нахожусь».

«У моей мамы была спутанность сознания и туман в голове. Она не могла делать
простые вещи, и что-то было не так. Нам пришлось оставаться с ней 24 часа в
сутки с друзьями и семьей, думая, что это что-то, что должно просто выйти из её
организма».

Джанни, которая в то время не знала, что Шерил была вакцинирована, сказала, что состояние её матери постепенно ухудшалось.

«Она перешла от состояния работать и заниматься обычными повседневными делами к
тому, чтобы еле-еле делать только основные вещи», — сказала Джанни. «До прививки
у неё была собственная квартира, и она каждый день работала торговым
представителем. Она готовила, занималась домашним хозяйством и занимала хорошее
положение в обществе.

Около 19 июня у Шерил снова сильно заболела голова, как объяснила её дочь.

«Несколько дней спустя я навестила её в больнице и не могла поверить своим
глазам», — сказал Джанни. «Она не могла ходить, говорила отрывистыми
предложениями, которые были непонятны, у неё были неконтролируемые движения
тела, её трясло, и она не могла сидеть на месте».

С каждым днём ей было всё хуже. «Это было ужасно, непонятно и по-настоящему
мучительно. Было тяжело осознавать, что её мозг больше ничего не контролирует»,
— рассказывает Джанни. «Первоначально врачи не нашли никаких отклонений у Шерил,
кроме слегка повышенного количества лейкоцитов. Но МРТ головного мозга выявила
признаки прионной болезни, что побудило врачей немедленно провести люмбальную
пункцию, которая исключила острую инфекцию, туберкулез, сифилис, рассеянный
склероз и другие заболевания».

Прионные болезни представляют собой семейство редких прогрессирующих
нейродегенеративных заболеваний, поражающих людей и животных. Прионные болезни
обычно быстро прогрессируют и всегда приводят к летальному исходу.

12 июля вторая люмбальная пункция показала положительный результат на CJD -
прионовую болезнь. Значение Тау-белка Шерил составило 38979 пг/мл, в то время
как спектр положительных пациентов с CJD составляет от 0 до 1149.

До постановки диагноза CJD.  В этот период «это было буквально похоже на то, как
будто кто-то ест её мозг», — говорит Джанни. «Вся трясясь от судорог, ей удалось
произнести слова «Какая глупость всё это».

Я спросила: «Мама, это вакцина?» и она ответила: «Да».

Джанни сказала, что была очень удивлена, когда узнала, что её мать была
вакцинирована, так как она происходит из семьи, где не признают вакцин. Она
считает, что, как и многие американцы, её мать чувствовала давление, чтобы
сделать вакцину из-за её работы и давления СМИ.

19 июля Шерил перевели в хоспис, где 22 июля она скончалась.

«Мы не знали, что делать», — сказала Джанни. «Это страшно. Мы не могли
остановить происходящее. Это похоже на быстродействующую деменцию. Это было
очень грустно, ужасно, безумно, чего её врачи никогда раньше не видели».
