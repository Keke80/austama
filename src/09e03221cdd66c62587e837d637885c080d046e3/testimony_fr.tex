Cette histoire est racontée par Valérie, la mère de Simone (19 ans), une
étudiante à l'université de Northwestern, elle reçu la première dose de Moderna
en avril. Elle se sentit très fatiguée après coup et développa une mauvaise
toux, mais pensait que c'était dû à ses allergies. Le temps passa et
soudainement nous étions en mai, et elle devait recevoir la seconde
dose. Cependant, elle ne se sentait toujours pas elle même, elle bloqua le
rendez-vous et y alla. Au cours des deux semaines qui ont suivi, elle
expérimenta des saignements de nez fréquents, une température très basse, des
irritations à la gorge, des ganglions enflés et un rythme cardiaque
irrégulié. Pendant cette période, elle visita la clinique universitaire trois
fois, où ils présumèrent qu'elle avait attrapé une sorte de virus et elle reçu
des médicaments antiviraux.

Le 16 mai, elle allait encore plus mal. Prise de vertige, sérieusement fatiguée
et impossibilité de marché, Simone a été admise à l'hopital ou elle et sa
famille ont appris que son coeur s'était arrêté de battre. Ses médecins ont
suspecté que c'était une myocardite, et qu'un “virus s'attaquait à son
coeur”. Une chirurgie en urgence pour insérer une pompe a été réalisée, mais
elle échoua et Simone fut placée sous respiration artificielle. Le 23 mai, les
docteurs ont informé ses parents que la transplantation de coeur était la
meilleure solution et qu'elle a la chance d'en recevoir un dans la soirée. A la
suite de la chirurgie, Simone continuait de se battre et devait rester sous
sédatif. Ses poumons avaient été sérieusement endommagés et les médicaments
immunodépresseurs qu'elle reçu pour tolérer la transplantation conduit à une
infection des poumons. Le 11 juin, les parents de Simones ont été appelés pour
lui dire un dernier au revoir. Simone décéda le vendredi matin.

«J'ai perdu ma fille unique. Je n'aurais jamais pensé que j'aurais du donner ma
fille pour le bien de la société. Je suspecte que c'était le vaccin. si ce n'est
pas le cas, il joua un rôle. Je ne savais pas qu'il y aurait un risque aussi
sérieux, si cela avait été dis clairement, je n'aurais jamais voulu que Simone
reçoive le vaccin.»

