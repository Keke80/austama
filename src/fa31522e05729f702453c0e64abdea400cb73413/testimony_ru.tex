56-летний Алекс потерял ногу из-за образования тромбов через две недели после
вакцинации AstraZeneca. Он получил свою первую дозу 20 марта 2021 года и был
госпитализирован 4 апреля после того, как потерял сознание у себя дома. Сгустки
крови были обширные, так что у врача не было другого выбора, кроме как
ампутировать ему ногу выше колена. Через двенадцать дней после вакцинации Алекс
пожаловался на то, что у него болят икры, «но я старался не думать об этом», -
сказал он. Боль не утихала в течение следующих нескольких дней, и 4 апреля Алекс
упал в обморок в своем доме. «Минутой раньше я гладил утюгом, и вдруг ноги
подкосились». Алекса доставили в больницу, где компьютерная томография
подтвердила наличие множественных тромбов в нижней части живота и в обеих
ногах. Хирурги смогли удалить сгустки из его живота, которые, к счастью, не
переместились в его печень или почки. «Я сделал вакцину, потому что хочу, чтобы
все как можно быстрее нормализовалось. Я думал, что единственный способ добиться
этого - это сделать вакцину. Я бы не хотел отговаривать людей от вакцины против
covid, но, если мы верно понимаем, то, что случилось со мной, бывает
редко. Несколько недель это было нереально тяжело. Я пережил самые ужасные дни в
моей жизни, но я все ещё жив, и мне нужно продолжать бороться и оставаться
позитивным. Это все, что я могу сделать ».

«Я спросил, могут ли они ампутировать мне ногу ниже колена, просто чтобы были
другие какие-то возможности протезирования», - сказал Алекс. «Но, к сожалению,
им пришлось ампутировать выше колена, так как все вены на ноге были негодны».

Алексу сказали, что ему понадобится как минимум год, чтобы снова стать на ноги с
протезом.

«Это изменило мой взгляд на всё», - сказал Алекс. «Это изменило мою жизнь. Я
потерял ногу и потерял возможность зарабатывать деньги. Я не смогу больше
никогда заниматься строительными лесами, и передо мной всё еще стоит задача
научиться ходить, но я думаю, что всё идёт к лучшему. До того, как все это
произошло, я не любил ничего больше, чем танцевать под Northern Soul. Я ужасный
танцор, так что это не большая потеря, но буду ли я танцевать когда-нибудь
опять? »

«Я считаю, что мне повезло. У меня есть семья и друзья, поддержка и любовь - и
много решимости. Важно быть благодарным за то, что у нас есть, и помнить, что
всегда есть кто-то несчастнее нас».

NoMoreSilence общался с шестью другими инвалидами в Великобритании, все из
которых напрямую связаны с вакциной AstraZeneca.

