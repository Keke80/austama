Gestorben an impfstoffinduzierter Thrombose und Thrombozytopenie (VITT).

Stephen hatte über ein kribbelndes Gefühl in seinem Arm geklagt. Zu seinem und
Charlottes Unglauben begann sich Stephens Arm von selbst zu bewegen. Das
verblüffte und verwirrte Paar rief sofort einen Krankenwagen, da Stephen
besorgniserregende Anzeichen eines möglichen Schlaganfalls zeigte. Als die
Sanitäter eintrafen, war Stephens Blutdruck alarmierend hoch, so dass er für
weitere Untersuchungen und Scans ins Krankenhaus gebracht wurde.

Mehrere Scans ergaben, dass sein Gehirn stark blutete und er einen sehr seltenen
zerebralen Schlaganfall erlitten hatte. Stephens Zustand verschlechterte sich
weiter, so dass er während der Scans im King's College Hospital einen
Krampfanfall erlitt.

Leider wurde Stephens Tod noch am selben Tag, dem 26. Januar, bestätigt. Doch
auch nach seinem Tod hat Stephen weiter Gutes getan, denn sein Wunsch, seine
Organe zu spenden, wurde erfüllt. Er wurde für einige Tage an die
lebenserhaltenden Maßnahmen angeschlossen, bis Spender gefunden waren.

Stephen war Spezialist für psychische Gesundheit und sollte demnächst eine neue
Stelle als leitender klinischer Psychologe am Great Ormond Street Hospital
antreten. Doch Stephens Zukunft wurde ihm auf verheerende Weise genommen.

Er hinterlässt seine Frau Charlotte und zwei kleine Söhne, Izaac und Elijah.
