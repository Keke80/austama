J'avais une maladie cardiaque et j'avais passé de nombreuses années à me refaire
une santé. On m'avait donné 6 mois à vivre en 2016 et on m'a implanté un
défibrillateur. Je suis paysagiste et, même si j'ai une soixantaine d'années,
j'étais capable de travailler 6 à 8 heures par jour en faisant un travail très
intensif. Je soulève des charges lourdes, je ratisse, je pellette, etc. De plus,
je promène mon chien tous les jours et je fais de courtes randonnées. En
général, mon cœur ne m'empêchait pas de faire tout ce que je voulais faire.

Le lendemain du vaccin Pfizer, j'ai commencé à avoir de l'arythmie, j'étais très
fatigué, faible, étourdi et j'avais l'impression d'avoir la grippe. J'ai manqué
quelques jours de travail. Je n'étais pas trop alarmée au début, je m'attendais
à ne pas me sentir bien pendant quelques jours. Mais je ne me suis jamais sentie
mieux, en fait, je me suis sentie encore plus mal au fil du temps.

J'ai appelé le service d'assistance téléphonique pour les vaccins et on m'a dit
que ce n'était pas dû au vaccin.

Le 16 juillet, mon cœur s'est arrêté, j'étais seule à la maison, mon
défibrillateur a relancé mon cœur et je me suis réveillée avec une cheville
cassée.

Depuis, j'ai subi de nombreux examens et mon cardiologue affirme qu'il n'y a
aucune raison médicale pour que mon cœur se soit arrêté ce jour-là. L'arythmie a
commencé le lendemain du vaccin Pfizer.

Je veux que les victimes des effets secondaires des vaccins soient reconnues,
soutenues, et non punies et mises au ban de la société.
