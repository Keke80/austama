Admis à l'hôpital du 1er mars 2021 au 4 avril 2021

Écrit par Adrian le 2 novembre 2021 :

Après le vaccin AstraZeneca, je ne me sentais pas bien, mais je m'y
attendais. J'ai donc essayé de continuer à vivre normalement, en pensant que les
effets disparaîtraient bientôt. Le 28 février 2021, je me suis senti très mal,
j'avais des douleurs dans tout le corps et de la fièvre. J'ai réservé un test
PCR Covid-19 pour ce jour-là et mes résultats sont arrivés le lendemain matin,
lundi 1er mars, négatifs.

J'ai contacté mon médecin généraliste qui m'a conseillé d'aller directement à
l'hôpital. Je me suis rendu à l'hôpital local de Grantham, où des analyses
sanguines ont montré que mon taux de plaquettes était de 7. J'ai donc été
transféré en quelques minutes à l'hôpital du comté de Lincoln, où des scanners
ont révélé la présence de plusieurs caillots sanguins. J'ai passé 34 jours à
l'hôpital de Lincoln et j'ai été nourri par un tube en raison d'un caillot de
sang dans ma veine porte provenant de l'intestin. Une ablation complète de
l'intestin a été envisagée et, le 8 mars 2021, ma femme a été appelée à
l'hôpital et on lui a dit que je ne survivrais peut-être pas. Heureusement, j'ai
survécu et j'ai toujours mon intestin. Mon hématologue m'a dit que l'une des
principales raisons pour lesquelles j'ai survécu était que j'étais auparavant en
bonne santé et que la quantité de médicaments nécessaires pour me maintenir en
vie aurait tué la plupart des gens.

Un compte rendu de la façon dont j'ai été traité par la communauté médicale : Au
début, les médecins étaient sceptiques quant à l'éventualité d'une réaction
indésirable au vaccin AstraZeneca, mais ils se sont ensuite ralliés à l'idée,
disant qu'ils devaient étudier toutes les autres causes possibles avant de me
diagnostiquer formellement une thrombose thrombocytopénie induite par le
vaccin. Mon hématologue et le personnel infirmier ont été superbes. Mon
hématologue m'a même téléphoné pour me communiquer les résultats de certains
tests alors qu'il était en vacances.

J'étais sous l'emprise d'un grand nombre de médicaments différents sous
perfusion, ce qui m'a fait perdre la tête pendant une grande partie de mon
séjour à l'hôpital. Je n'étais pas tout à fait sûr de ce qu'il se passait, mais
j'essayais simplement de survivre. On m'avait installé une ligne de piqûre, par
laquelle on me nourrissait. On m'a administré des doses extrêmement élevées de
stéroïdes et d'anticoagulants et on m'a fait des transfusions de sang et de
plaquettes.

Lorsque l'infirmière m'a poussé par les portes de l'hôpital le 4 avril 2021,
j'ai poussé un énorme soupir de soulagement, car j'avais réussi à sortir de
l'hôpital et à survivre, mais ce n'était qu'une partie de la bataille
gagnée. Aujourd'hui encore, la bataille continue, et il semble que ce soit pour
très longtemps.

Je dois encore prendre cinq lots de médicaments et on m'a dit que je devais
m'attendre à devoir prendre des anticoagulants pour le reste de ma vie.

Un message pour les autres sur l'impact que cela a eu sur votre vie :

J'étais un homme de 47 ans très en forme, en bonne santé et actif, un non-fumeur
qui buvait très peu d'alcool, un marathonien pouvant courrir 4 heures et un
ancien arbitre de football semi-professionnel. Aujourd'hui, je ne peux même pas
vivre de façon indépendante, car je suis dépendant de ma famille pour vivre au
quotidien. Je suis essoufflé tous les jours. Même m'habiller me coupe le souffle
et je dois m'asseoir pour récupérer avant de pouvoir faire quoi que ce soit
d'autre. Je ne peux pas marcher normalement, je traîne les pieds comme un vieil
homme parce que mes jambes sont si faibles que je ne peux pas marcher avec une
foulée normale.

J'ai dû acheter un scooter de mobilité pour essayer de me donner plus
d'indépendance. Bien souvent, je ne prends pas de bain parce que je suis trop
épuisé. J'ai l'impression d'exister, mais de ne plus vivre, car la vie que je
connaissais m'a été enlevée.

Je suis maintenant traitée pour le syndrome de fatigue chronique et j'attends de
voir le consultant en pneumologie de l'hôpital de Lincoln.

Moi-même et ma famille avons été affectés physiquement, mentalement et
financièrement. Il y a une réelle possibilité que nous perdions notre maison à
un moment donné dans le futur car je n'ai pas été capable de travailler depuis
le 26 février 2021 et je suis maintenant à un moment de ma vie où je commence à
penser que je ne serai peut-être jamais assez bien pour travailler à nouveau.

Nous ne pouvons même pas faire une sortie en famille car je ne peux pas marcher
plus de deux cents mètres, certains jours je ne peux même pas marcher cinquante
mètres. Je suis constamment fatigué physiquement et mentalement, la vie est
débilitante et certains jours, je souhaite ne pas avoir survécu car je ne veux
pas vivre le reste de ma vie comme ça.

J'ai fait “ce qu'il fallait”. J'ai reçu le vaccin AstraZeneca Covid-19, et
maintenant ma famille et moi-même avons vu nos vies basculer. Le gouvernement
devrait s'occuper des personnes qui ont été gravement affectées par le vaccin
Covid afin que nous puissions vivre sans nous soucier de payer notre hypothèque
et nos factures.

Je ne suis pas “anti-vaccins”, j'ai fait “ce qu'il fallait” comme on me l'avait
demandé, et je souffre depuis !
