Amy, la mère d'Abby, raconte son histoire :

Je pense qu'avec tout ce qui se passe, je dois m'exprimer. Ma fille avait prévu
un voyage de fin d'études à Hawaï avec son ami et la famille de son ami. À
l'époque, Hawaï avait un protocole Covid très strict si vous n'étiez pas
vacciné. En dépit de mon bon sens, je lui ai permis de recevoir le vaccin
Pfizer. Le 31 mars, elle a reçu sa première injection.

Le lendemain, elle ne se sentait pas très bien, mais elle a insisté pour aller
travailler. Cet après-midi-là, son employeur nous a appelés pour nous dire
qu'Abby s'était évanouie et qu'elle tremblait de façon incontrôlable. Les
tremblements se situaient sur le côté droit de son corps, le côté opposé à celui
où elle avait reçu l'injection.

On l'a emmenée aux urgences où on l'a branchée en intraveineuse et on lui a
donné deux médicaments différents. On lui a dit qu'elle avait eu une réaction
neurologique indésirable au vaccin.

En raison du manque de recherches ou d'études sur les effets du vaccin, les
médecins n'ont pas été en mesure de nous donner beaucoup d'informations. Il se
peut que cela disparaisse au bout de quelques jours, mais il est fort probable
que le deuxième vaccin provoque une réaction plus grave, mais personne ne le
sait “, ont-ils dit.

Après 2 ou 3 jours, les tremblements n'étaient plus aussi graves. Cela fait
maintenant 6 mois et les tremblements sont toujours perceptibles dans son bras
droit lorsqu'elle essaie de rester immobile ou de tenir un objet, et ils sont
encore plus perceptibles après un exercice. Quelques semaines plus tard, nous
avons reçu un remboursement complet de Kaiser pour la visite aux urgences,
puisque le vaccin en était la cause. Nous avons trouvé cela étrange !

Abby fait partie de l'équipe de pom-pom girls du Lycée Mt. SAC et tous les
athlètes ont été informés qu'ils devaient désormais être entièrement
vaccinés. Abby a jusqu'au 15 octobre ou elle sera renvoyée de l'équipe. Elle a
fait une demande d'exemption médicale auprès de Kaiser en raison d'un cas
documenté de réaction indésirable au vaccin et elle a été refusée ! On lui a dit
qu'il existait d'autres vaccins qu'elle pouvait essayer. (Elle est une
expérience !)

Nous entendons tous dire que le vaccin est sûr… À MOINS QUE vous ne soyez l'une
des victimes malchanceuses ! Je ne permettrai pas que nos enfants soient des
expériences ! Je vais me lever et me battre. Je suis le parent de mon enfant,
pas le gouvernement !

Je veux informer autant de personnes que possible de ces dangers.

