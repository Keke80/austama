Isabelli, une étudiante de 16 ans, est décédée 8 jours après avoir reçu le
vaccin Pfizer COVID-19.

Suite au décès d'Isabelli, le ministère brésilien de la santé a suspendu la
vaccination contre le COVID-19 des personnes âgées de 12 à 17 ans.

Isabelli a reçu son vaccin Pfizer le 25 août. Elle est décédée de problèmes
cardiaques le 2 septembre. Isabella a présenté des symptômes tels que des
étourdissements, des maux de tête, un essoufflement et une somnolence un jour
après la vaccination. Le dimanche 29, sa mère (Cristina) l'a emmenée à l'hôpital
après qu'elle se soit plainte de picotements dans le corps. “Le médecin traitant
a dit que ces symptômes pouvaient être psychologiques en raison du vaccin et
qu'elle pourrait sortir de l'hôpital. Avant de quitter l'hôpital, Isabelli a
perdu connaissance”, a déclaré la mère.

La jeune étudiante a été transférée dans un hôpital de São Paulo, où l'on a
constaté qu'Isabelli présentait des taux extrêmement bas de globules rouges et
d'hémoglobine, et qu'une transfusion de huit poches de sang était
nécessaire. Après une crise d'épilepsie, elle a été emmenée à l'unité de soins
intensifs. Elle est décédée le 2 septembre.

Cristina a déclaré qu'Isabelli avait reçu le vaccin parce qu'il était
obligatoire pour aller à Disney, ce qui était son rêve. “Elle a toujours été
très dévouée. Nous n'étions que tous les deux. Elle était ma vie, ma raison de
vivre”, a déclaré Cristina.

Selon Cristina, “le certificat de décès montre que sa fille est morte de trois
causes possibles : choc cardiogénique, infarctus aigu du myocarde (crise
cardiaque) et anémie sévère”.

En raison de réactions indésirables au vaccin destiné aux personnes âgées de 12
à 17 ans, le ministre brésilien de la santé, Marcelo Queiroga, a suspendu la
vaccination de ce groupe d'âge et une enquête sur sa mort soudaine a été
ouverte.

