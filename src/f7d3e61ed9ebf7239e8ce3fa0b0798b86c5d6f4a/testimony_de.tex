

Ich bin Krankenschwester auf der Covid PCU(kritische Pflegeabteilung). Ich habe mich im Juli 2020 freiwillig gemeldet, um auf dieser Station zu arbeiten, und arbeite dort weiterhin mit Covid-Patienten. Ich wollte den Impfstoff. Ich habe geglaubt, dass er die Sache beenden würde.

Die erste Impfung machte mich sehr krank und müde. Zuerst waren stieg die Herzfrequenz und ich bekam Gesichtsrötungen, die ich ignorierte. Heiligabenf fühlen ich mich, als wäre ich von einem Zug überfahren worden. Fünf Tage lang fühlte ich mich ausgelaugt und hatte Schmerzen Dann ging es mir besser. Aber von diesem Zeitpunkt an hatte ich mehr Gelenkschmerzen als zuvor.

Nach der zweiten Impfung fühlte sich mein Gesicht taub und kribbelig an. Das gleiche Gefühl hatte ich gelegentlich bei meinen Migräneanfällen. An diesem Tag hatte ich ein Face-Timing mit meiner Mutter, die auch Krankenschwester ist. Sie war überzeugt, dass ich die Bellsche Lähmung hatte. Ich habe ihr wiedersprochen.

Zwei Tage später kehrten diese Symptome mit voller Wucht zurück und verschwanden nicht mehr. Ich bekam Elektroschocks in der linken Schläfe, die hinter mein Auge ausstrahlten. Ich bekam Kopfschmerzen, die durch keine Behandlung verschwanden. Ich nahm rezeptfreie und verschriebene Schmerzmittel, aber sie halfen nicht. Das Brennen in meinem Gesicht war wahnsinnig.

Meine Symptome wurden immer schlimmer. Ich ging zur Behandlung in die Notaufnahme und man sagte mir, ich hätte wahrscheinlich eine Gürtelrose im Auge.

Ich ging zu einem Augenarzt, der diesen Befund widerlegte und sagte: Nein, Sie haben Trigeminusneuralgie. Ich musste auf einen Termin warten. Bei meinem ersten Termin wurde mir Topamax und Gabapentin verschrieben.

Ein MRT wurde verschrieben. Die Medikamente waren nicht gut für mich. Topamax hat mich völlig umgehauen! Bei der MRT wurde eine intrakranielle Hypertonie festgestellt. Wow!!! Wo kommt das denn her? Ich war erschrocken!

Während ich auf den Termin beim Neurologen wartete, verursachte das Gabapentin eine Niereninfektion und machte mich extrem krank. Meine Symptome wurden so schlimm, dass ich Schwindel, Übelkeit und verschwommenes Sehen bekam und 3 Monate FMLA(Arbeitsnehmerschutz im Krankheitsfall) in Anspruch nehmen musste.

Der Neurologe verschrieb mir Diamox 1000 mg, Lasix und verschiedene Migränemittel. Ich kann zwar besser arbeiten als zu der Zeit, als ich FMLA beantragen musste,

Aber ich habe immer noch ständige Kopfschmerzen, Sehstörungen und manchmal eine verschwommene Wahrnehmung. Ich warte auf einen Termin mit einem anderen Neurologen, der sich vielleicht um eine bessere Behandlung mit hoffentlich besseren Aussichten bemüht.

Ich möchte wieder so leben wie früher.
