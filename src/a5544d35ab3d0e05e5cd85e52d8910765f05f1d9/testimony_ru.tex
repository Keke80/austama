
История Матильды рассказана её собственными словами:

Я получила первую дозу Пфайзера во вторник, 25 мая. В среду вечером и в четверг
моя левая рука стала фиолетовой (укол был на левой руке) и эту же руку
парализовало. Я не могла пошевелить рукой, она была твёрдая и тяжёлая.

На следующий день боль в руке исчезла, но начала болеть левая нога. Боль была
терпимой, но я хромала. В понедельник утром я проснулась и не смогла наступить
на пятку, это было очень больно. Не могла пошевелить пальцами ног, они
занемели. Меня отвезли в больницу, на неделю прописали морфин и кетопрофен, но
ничего не помогло. Я была госпитализирована на 10 дней, в течение которых мне
сделали компьютерную томографию поясничного отдела спинного мозга, затем
ЭМГ(электромиографию), которая подтвердила радикулит.

Они пытаются увеличить дозу морфина, а затем кортикостероидов, но мне ничего не
помогает. Я хромаю, а когда сижу или стою - нет устойчивости.

Я вышла из больницы после 5 дней курса иммуноглобулинов и люмбальной
пункции. Через неделю дома у меня были ужасные побочные эффекты: рвота, сильная
головная боль, утомляемость. Я не могла ходить.

Через неделю меня снова госпитализировали, чтобы пройти ещё одну терапию ЭМГ и
иммуноглобулином. У меня была большая потеря крови (сгустки). Мы снова
попробовали Акупан и Трамадол… затем вторая ЭМГ подтвердила, что радикулит всё
ещё присутствует.

Сейчас, через 2 месяца после инъекции, я принимаю противоэпилептики,
нейролептики, кетопрофен и триптан два раза в день, потому что головные боли
такие сильные, просто непереносимые.

У меня нет примерной даты излечения, так как радикулит после вакцинации вообще
неизвестен.

Я сделала прививку, чтобы получить рабочий контракт в медико-социальном секторе,
а сегодня я не могу работать. Я больше не пойду туда, так как мой контракт
истекает через 3 недели. Это ужасно!

«The Gray's press» также опубликовала случай Матильды:

«Когда Матильда Харисменди работала по контракту, её менеджер «настоятельно
посоветовал» ей пройти вакцинацию, чтобы сохранить работу, что она и сделала. В
результате сегодня её повседневная жизнь превратилась в ад!

Прошли те времена, когда Матильда Харисменди могла ходить по магазинам, гулять
или ездить на велосипеде с детьми вокруг озера. Делать спокойно покупки или
просто посещать своих родителей. Действия, которые многим кажутся безобидными,
стали настоящим испытанием для 27-летней Фротеенн, которая, против своего
желания, с мая попрощалась со своими повседневными делами, а теперь и со своей
работой».
