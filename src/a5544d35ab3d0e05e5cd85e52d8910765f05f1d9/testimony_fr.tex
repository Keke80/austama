L'histoire de Mathilde racontée avec ses propres mots:

J'ai reçu ma première dose de Pzifer le mardi 25 mai. Mercredi soir et jeudi, ma
main gauche est devenue violette (vaccin au niveau du bras gauche) et une
paralysie du même coté. Je ne pouvais pas bouger mon bras et il était dur et
lourd.

Le lendemain, les douleurs au bras ont disparu mais ma jambe gauche a commencé à
me faire souffrir. La douleur était supportable mais je boitais. Lundi matin, je
me lève et je ne peux plus poser le talon, c'était super douloureux. Je ne peux
pas bouger mes orteils, ils sont gelés. On m'a emmené à l'hôpital et on m'a mis
sous morphine et sous Ketoprofen pendant une semaine, mais rien n'y fait. Je
suis hospitalisée depuis 10 jours, pendant lesquels j'ai passé un scanner
cérébral lombaire et médullaire, puis un EMG qui confirme une radiculite.

Ils essaient d'augmenter les doses de morphine puis de corticoïdes mais rien ne
me soulage. Je souffre de boiter en position assise, couchée et debout - je n'ai
aucune stabilité.

Je sors de l'hôpital après avoir reçu 5 jours d'immunoglobulines et une ponction
lombaire. Une semaine plus tard, j'ai eu de terribles effets secondaires chez
moi: vomissements, mauvais maux de tête, fatigue. Je ne pouvais pas marcher.

De nouveau, j'ai été hospitalisée une semaine plus tard, pour refaire un EMG et
un traitement par immunoglobulines. J'ai eu de grosses pertes de sang
(caillots). Nous avons à nouveau essayé l'Acupan et le Tramadol… puis un second
EMG confirmant que la Radiculite est toujours présente.

Aujourd'hui, 2 mois après l'injection, je suis sous anti-épileptique,
neuroleptique, kétoprofène et triptan 2 fois par jour tant mes maux de tête sont
violents et difficiles à supporter.

Je n'ai pas de date de guérison puisque la radiculite post-vaccinale est
inconnue.

Je me suis fait vacciner pour obtenir un poste sanitaire et social et
aujourd'hui j'arrête le travail. Je n'irai plus puisque mon contrat s'arrête
dans 3 semaines. Je suis en colère et dégoûtée !

“The Gray's press” a également fait écho du cas de Mathilde :

“Lorsque Mathilde Harismendy était en CDD, sa responsable lui a “fortement
conseillé” de se faire vacciner pour conserver son emploi, ce qu'elle a
fait. Résultat, aujourd'hui, son quotidien est devenu un enfer !

Fini le temps où Mathilde Harismendy pouvait faire ses courses, se promener ou
faire du vélo avec ses enfants autour du lac. Faire ses courses en toute
confiance ou simplement rendre visite à ses parents. Des activités qui semblent
anodines pour beaucoup de gens, sont devenues un calvaire pour la Frotéenne de
27 ans qui a dit adieu, bien malgré elle, à ses gestes quotidiens depuis le mois
de mai, et maintenant à son travail.”

