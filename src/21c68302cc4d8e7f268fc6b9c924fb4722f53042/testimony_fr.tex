\testimony{Julian Schofield}
{Vancover, Canada}
{Non communiqué}
{johnson \& johnson}
{13 juillet 2021}
{paralysis, encéphalomyélite disséminée aiguë, inflammation du cerveau et de la moelle épinière}
{picture.jpg}
{https://nomoresilence.world/johnson/julian-schofield-johnson-johnson-severe-paralysis/}
{

Angela raconte l'histoire de son mari:

La paralysie de Julian commença 12 jours après sa vaccination. Julian est
actuellement paralysé jusq'à la taille.

Le 25 juillet, julian commença à ressentir des picotement ainsi que des
engourdissement dans son pied gauche. Cela à progresser rapidement jusqu'à son
pied gauche et commença au niveau de son pied droit pour voyager jusqu'à sa
jambe droite. Dans les deux heures qui ont suivit, Julian ne pouvait plus
marcher ni tenir debout. Nous avons accouru à l'hopital regional de Penticton.

Le neurologue de Penticton s'est inquiété du fait que Julian avait développé une
forme myélinique d'encéphalomyélite aiguë disséminée post-vaccinale (ADEM).

Cet état est caractérisé par une bref mais généralisé attaque inflammatoire dans
le cerveau et la moelle épinière qui endommage la myéline, la couche protectrice
des fibres nerveuses.

Julian a des gonflements au niveau de ses vertèbres T7 et T8 ce qui a causé sa
paralysie.

Après deux passages sous stéroïdes et une thérapie IVIG, quelques gonflements
ont diminués et nous espérons que Julien retrouvera sa mobilité. À cette date,
la mobilité de Julian n'est pas revenue.

Nous avons espoir de ramener Julian à la maison le 22 octobre. Cette date
marquera pratiquement ses 12 semaines d'hospitalisation.

De la part de Julian, Emma, Thea et moi-même, nous vous partageons nos meilleurs
sentiments, merci pour tout l'amour et le support que vous nous avez partagé
pendant cette période difficile.

}
