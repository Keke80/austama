Jennifer, eine 37-jährige Mutter von zwei Kindern ohne gesundheitliche
Vorbelastung, hat beim Impfprogramm mitmachen und sich eine zweites Mal impfen
lassen. Sie hat jedoch von ihren Ärzten widersprüchliche Ratschläge erhalten,
nachdem sie seit der ersten Pfizer-Impfung, Anfang Juli 2021, unter chronischem
und schwächendem Tinnitus (Ohrensausen) und Hyperakusis
(Geräuschempfindlichkeit) leidet.

Sie ist besorgt, dass die zweite Dosis ihren Zustand verschlimmern könnte. Ihr
Hausarzt und ein Hals-Nasen-Ohren-Arzt, die sie konsultiert hat, haben ihr
gesagt, dass dies durchaus passieren kann.

Jennifer sagt, sie habe mit sieben weiteren Ärzten, Beratern und anderen
Medizinern gesprochen, und alle, bis auf einen, hatten ihr entweder gesagt, sie
könnten ihr nicht raten, ob sie die zweite Dosis nehmen solle oder sie hatten
ihr davon abgeraten.

Ihre Ärzte haben ihr mitgeteilt, dass der Zustand nun möglicherweise dauerhaft
sei.

Sie hat an den Gesundheitsminister Stephen Donnelly geschrieben und um mehr
Unterstützung für Menschen in ihrer Situation gebeten.

Sie sagte, sie habe von der HSE keine Unterstützung erhalten und sei ``in der
Luft hängen'' gelassen worden. Sie wolle die zweite Impfung aus
Sicherheitsgründen für sich und ihre Familie.  Einerseits hat sie Angst, dass
eine Dosis nicht ausreichend ist, andererseits ist sie besorgt, dass sich ihr
Tinnitus verschlimmern könnte, sollte sie die zweite Impfung erhalten.

``Ich bin eine entschiedene Befürworterin der Impfung und jeden Tag dankbar für
die Vorteile des medizinischen Fortschritts'', sagte sie, aber sie hat das
Gefühl, dass man sie ``russisches Roulette'' spielen lässt, da sie das Risiko
einer Ansteckung mit Covid nicht unterschätzt, jedoch eine Verschlimmerung ihres
Zustandes befürchtet.
