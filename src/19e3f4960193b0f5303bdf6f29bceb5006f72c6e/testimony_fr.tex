Une heure après avoir reçu le vaccin AstraZeneca, j'ai ressenti une pression
extrême à la tête et j'ai commencé à développer des symptômes neurologiques de
brûlures et de douleurs musculaires, ainsi qu'une accélération du rythme
cardiaque. Je me suis sentie malade comme si j'avais une grippe très bizarre
pendant des semaines, avec un mal de tête constant. La troisième semaine, j'ai
ressenti de violentes douleurs aux côtes et à la poitrine, ce qui a nécessité
une visite aux urgences.

Ils ont fait une radiographie et m'ont renvoyée chez moi. La douleur a continué,
ainsi que des crampes aléatoires dans les jambes et une sensation
d'essoufflement. Six semaines plus tard, j'ai eu une période étrange où j'ai
failli m'évanouir, avec des maux de tête bizarres et une pression accrue dans la
tête. Ma vision est devenue bizarre pendant cette période et j'ai eu peur.

Ma vue était maintenant diminuée d'un œil. J'ai aussi commencé à avoir de fortes
fourmillements dans les pieds et les mains. Mes jambes et mes bras s'endormaient
aussi souvent. Deux mois et une semaine après, j'ai eu une réaction
anaphylactique deux jours de suite, nécessitant des visites à l'hôpital. Un jour
après, j'ai eu une fibrillation auriculaire. J'étais sous stéroïdes à court
terme et sous de grandes quantités d'antihistaminiques.

L'hôpital a soupçonné que j'avais développé des allergies alimentaires. J'ai
également montré des signes d'infection inconnue dans mes analyses de sang et
j'ai développé un diabète de type 2 (en janvier, je ne l'avais pas). À la
mi-mai, j'ai développé un gonflement des glandes et des problèmes
d'amygdales. J'ai commencé à perdre ma voix. J'avais constamment peur de
m'étouffer, car même avaler devenait difficile. Je suivais un régime pauvre en
histamine et j'avais du mal à trouver des aliments qui ne me fassent pas
réagir. J'ai également commencé à réagir à tout ce qui m'entourait, à développer
de nouvelles éruptions cutanées et mes cheveux ont commencé à tomber. J'ai
ensuite eu des problèmes buccaux étranges qui ont été suggérés comme étant le
syndrome de la bouche brûlante. J'ai demandé à voir un ORL, un immunologiste, un
neurologue, un rhumatologue, bref, toute personne susceptible de m'aider, mais
sans succès. Six mois et demi plus tard, après avoir vu un dentiste et appelé
plusieurs médecins généralistes, j'ai consulté un ORL qui a constaté des
problèmes de larynx et des écoulements post-nasaux, mais qui ne s'est pas
inquiété de mes problèmes, estimant qu'ils n'étaient pas “de nature
sinistre”. Sept mois plus tard, j'ai payé pour voir un spécialiste qui s'occupe
du syndrome d'activation des mastocytes, qui m'a ensuite été diagnostiqué.

Ma vie a radicalement changé depuis le vaccin AstraZeneca. Je me bats pour
survivre chaque jour, en me demandant si tout ce que je mange, bois ou respire
va déclencher une anaphylaxie.

Je me suis battue pour obtenir une aide médicale depuis mon vaccin. Je me suis
sentie ignorée et rabaissée. J'ai été mal traitée et considérée comme hystérique
par un ambulancier, dans ma propre maison, parce que je faisais une crise de
fibrillation auriculaire. J'ai été ignorée toute la nuit à mon hôpital pour
attendre dans une salle d'entreposage sale alors que je faisais une réaction
anaphylactique. Les seuls soins que j'ai reçus sont ceux d'une minorité de
médecins généralistes et de mon spécialiste du MCAS. J'ai eu de nombreux
“diagnostics temporaires” suggérés depuis le vaccin, y compris divers types de
migraine, de neuropathie, de dysphagie, d'amygdalite, etc. On m'a maintenant
diagnostiqué un MCAS et un diabète de type 2.

J'ai signalé ma réaction au système britannique Yellow Card, à AstraZeneca, et
bien sûr mon médecin généraliste aisnsi que l'hôpital local sont au courant de
ce qui s'est passé. J'ai reçu une réponse médiocre de la part d'AstraZeneca et
aucune de la part du système.

Il n'est pas juste que le corps médical nous rejette, que l'on pense que le
vaccin est responsable ou non, nous avons toujours le droit d'être soignés. Nous
sommes maltraités, on nous envoie des gaz et nos vies sont détruites parce que
nous le faisons. On nous accuse d'être anti-vaccins ou d'être fous parce que
notre corps n'a pas réagi de façon “normale”.
