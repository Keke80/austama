\testimony{Shandrille  Dall'Agnese}
{France}
{19 ans}
{Pfizer/BioNTech}
{22 juin 2021}
{Thrombophlébite cérébrale}
{picture.jpg}
{https://nomoresilence.world/pfizer-biontech/shandrille-dallagnese-pfizer-severe-adverse-reaction-aged-19/}
{

L'histoire de Shandrille, 19 ans, nous est racontée avec ses propres mots :

Je m'appelle Shandrille et j'ai 19 ans. J'ai été vaccinée le 22 juin avec la
première dose du vaccin Pfizer, et une semaine plus tard, j'ai eu des maux de
tête qui ne se sont jamais arrêtés.

Le 2 juillet, je suis allée aux urgences et, après avoir passé des scanners, le
médecin m'a dit que j'avais une thrombophlébite cérébrale (forme rare d'accident
vasculaire cérébral) et j'ai été hospitalisée pendant 10 jours.

Le lendemain de ma sortie de l'hôpital, j'ai commencé à avoir des problèmes
oculaires. Je suis donc retournée aux urgences le 14 juillet, et après plusieurs
examens, le médecin m'a dit que j'avais une complication lié à
l'AVC. L'ophtalmologiste m'a expliqué que j'avais trop de pression dans mon
crâne ce qui a conduit à un œdème papillaire et à un “arrêt temporaire” d'un
nerf optique de l'œil gauche.

Le 22 juillet, le neurologue m'a fait une ponction lombaire pour diminuer cette
pression. Je reste sous comprimés pour faire baisser la pression et je suis
également sous anticoagulants.

À la suite de la ponction, ma vision s'est améliorée mais n'est pas encore
totalement rétablie. J'ai encore de nombreux examens prévus, et il me reste
encore plusieurs mois avant d'espérer revenir à 100\%.

Tous mes tests de recherche de problèmes de coagulation sont négatifs. La seule
explication plausible reste donc l'immunisation par le vaccin !

}
