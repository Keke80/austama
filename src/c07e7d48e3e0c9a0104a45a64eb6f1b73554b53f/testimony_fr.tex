L'histoire de Kerry, 49 ans, racontée par sa nièce Jodie :

Nous voulons partager avec vous l'histoire de ma tante Kerry qui vit au
Royaume-Uni.

Elle a reçu sa première dose d'AstraZeneca le 1er avril 2021 et a commencé à
présenter des symptômes le 8 avril 2021. Elle a commencé à ressentir de fortes
douleurs à la tête et aux yeux. Elle a appelé le médecin qui lui a conseillé de
passer un examen de la vue, mais elle n'a pu le faire qu'une semaine plus
tard. Elle a dit à son médecin qu'elle avait pris l'AstraZeneca une semaine
auparavant, mais il n'a rien dit de plus.

Le vendredi, comme elle n'allait pas mieux, elle a fait un test Covid, qui était
négatif.

Le samedi, comme elle n'allait toujours pas mieux, elle a appelé le 111, et le
service des urgences a rappelé dans les 10 minutes en disant qu'elle devait être
hospitalisée au plus vite.

Elle a alors passé un scanner qui a révélé la présence d'un caillot de sang dans
une veine de sa tête. Ses plaquettes n'étaient plus que de 19 et on essayait de
les augmenter assez lentement pour ne pas provoquer d'autres caillots, mais cela
a malheureusement provoqué une hémorragie cérébrale qui a entraîné une attaque
massive. Elle a subi des lésions cérébrales et a perdu la vue de l'œil gauche.

Elle a ensuite subi une intervention chirurgicale visant à lui retirer une
partie du crâne pour réduire le gonflement. On nous a dit qu'elle avait moins de
40\% de chances de s'en sortir. Heureusement, elle s'en est sortie, mais elle a
malheureusement perdu l'usage de son côté gauche. Pour l'instant, les progrès
sont extrêmement lents. Nous attendons que le gonflement diminue avant qu'ils
puissent remettre en place la partie du crâne qu'ils ont enlevée.

Elle suit des séances de kinésithérapie du lundi au vendredi, mais elle ne peut
faire qu'une heure par jour, car cela la fatigue énormément.

Elle était une personne de 49 ans heureuse et en bonne santé, mais maintenant
nous ne savons pas ce que l'avenir lui réserve.

