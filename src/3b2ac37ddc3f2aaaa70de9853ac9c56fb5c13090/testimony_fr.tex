Eva raconte son histoire :

Ma réaction n'a pas été immédiate. J'ai eu des frissons, des engourdissements et
des picotements environ 10 jours après ma deuxième injection.

Peu de temps après, j'ai développé une pression dans les oreilles et la tête qui
s'est aggravée. J'ai d'abord attribué tout cela à une infection chronique des
sinus, car je souffrais auparavant d'otites et d'allergies.

J'ai ensuite développé un tic incessant des paupières, des palpitations
cardiaques insensées, une tachycardie, des douleurs thoraciques sévères et un
essoufflement.

Comme tous mes symptômes persistaient, j'ai fini par consulter mon médecin
traitant en mai, qui m'a fait passer des analyses de sang standard et un
électrocardiogramme, et m'a orientée vers un ORL et un cardiologue. Je ne savais
toujours pas quoi faire.

Puis la douleur débilitante dans mes jambes a commencé. Elle était si atroce que
je ne pouvais pas prendre de douche, me lever ou marcher. Peu de temps après,
elle s'est propagée à la partie supérieure de mon corps et s'est accompagnée
d'une pression nauséeuse à la tête qui m'a fait me recroqueviller en position
fœtale pendant des jours.

Je me suis rendu aux urgences lorsque j'ai réalisé que je perdais l'équilibre et
les sensations dans mes deux jambes. Je craignais une paralysie.

Après qu'une IRM et un scanner aient écarté la possibilité d'une SEP, j'ai été
renvoyée chez moi pour faire face seule à la douleur.

J'ai continué à développer de nombreux nouveaux symptômes plusieurs mois après,
sans réponse ni traitement.

J'étais en parfaite santé avant ma deuxième dose de Pfizer, et j'ai su que ce
n'était pas une coïncidence après que tous mes tests soient revenus normaux.

Je n'ai plus de vie. Ma douleur a atteint un point où elle est constante et
débilitante, de sorte que je ne peux plus faire des choses simples comme
marcher, nettoyer, aller au magasin ou m'occuper de mon fils. Je suis à la merci
de ma famille.

