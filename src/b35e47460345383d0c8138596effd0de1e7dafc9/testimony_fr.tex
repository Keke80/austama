Ben, un Australien vivant en Nouvelle-Zélande, raconte son histoire en date du
30 novembre 2021 :

Je veux que les gens sachent que cela arrive à d'autres personnes. Si cela vous
est arrivé, vous devez vous sentir à l'aise pour en parler afin que les
professionnels de la santé soient plus conscients.

J'aimerais dire que je comprends que chacun a ses propres opinions. Je ne
souhaite pas détourner qui que ce soit de son propre choix. J'exprime simplement
mon diagnostic/expérience actuel après avoir reçu ma première dose du vaccin
Pfizer.

Après l'injection, j'ai ressenti des symptômes semblables à ceux de la
grippe. Maux de tête, sueurs froides et fatigue (qui sont des effets secondaires
courants). Puis j'ai commencé à ressentir une douleur en forme de coup de
poignard au milieu, légèrement à gauche du centre de ma poitrine. Elles étaient
sourdes mais bien présentes. J'avais également une blessure sur mon genou
gauche, ce qui a provoqué un gonflement important de ma jambe. Un gonflement
anormal, car je n'avais jamais eu cela auparavant.

Rien de tout cela ne m'a inquiété jusqu'à ce que mon genou commence à enfler
sérieusement et que mes douleurs thoraciques deviennent fréquentes. Miki a
appelé le service des soins et on m'a dit d'aller directement à l'hôpital. Nous
nous sommes précipités et avons fait tous les tests et les scans
nécessaires. Tout était clair, mais on m'a dit que mon cœur était peut-être
irrité par le vaccin Pfizer. On m'a prescrit des médicaments, principalement
pour le gonflement du genou, et je suis rentré chez moi, pour ne revenir que si
les symptômes s'aggravaient.

Trois jours plus tard, dans l'après-midi, j'ai commencé à ressentir une douleur
constante à la poitrine et un essoufflement, surtout en position couchée. Après
quelques heures passées avec la douleur, la jambe toujours enflée, je suis
retournée à l'hôpital. J'ai passé de nombreux autres tests qui se sont avérés
concluants et j'ai été examinée par un autre médecin. J'ai expliqué tous les
symptômes dont je souffrais et on m'a dit que mon cœur souffrait d'une
péricardite due au vaccin. J'ai reçu un autre lot de médicaments et j'ai été
renvoyée chez moi avec cette douleur thoracique anormalement aiguë.

Le 6e jour, j'ai vu mon médecin généraliste qui a refait les tests et a constaté
une légère anomalie dans les résultats de mon ECG, confirmant ainsi que le
diagnostic était correct. Il m'a de nouveau expliqué que si les symptômes
s'aggravaient ou si les veines commençaient à remonter dans ma jambe, je devais
demander une aide médicale.

Le lendemain matin, ma jambe ne pouvait pas supporter le poids de mon corps et
j'avais une varice qui sortait de derrière mon genou, remontant sur 5 cm et
dépassant d'un centimètre. Nous avons immédiatement appelé l'ambulance et, à
leur arrivée, la veine était redevenue normale (malheureusement, dans la
panique, nous n'avons pas pris de photo).

Une fois de plus, des machines et des tests ont été effectués, qui n'ont rien
donné. Les ambulanciers ont prévenu l'hôpital local qui leur a simplement dit :
“Si la veine est partie et que ses signes vitaux sont clairs, nous n'avons pas
besoin de le voir”. Les ambulanciers nous ont semblé inquiets et nous ont
conseillé personnellement de nous rendre dans un hôpital plus grand. Nous avons
donc pris la voiture pour un voyage matinal de 2 heures et demie.

En arrivant aux urgences, on a effectué des tests qui étaient tous bons. Un
docteur X s'est alors assis et m'a demandé pourquoi j'étais là. J'ai expliqué
“ma douleur à la poitrine et ma jambe enflée”. Elle m'a expliqué qu'elle avait
examiné tous mes examens et que tout était normal. Alors que j'essayais
d'expliquer l'aggravation de mon état avec de nouveaux symptômes, par exemple la
veine dans ma jambe et l'œdème de Quincke (signes de caillots sanguins), elle
m'a coupé la parole en disant “vos tests sont bons, vous n'avez rien, vous
souffrez probablement de ces effets secondaires dus à l'anxiété et je ne peux
rien faire pour vous”, ignorant complètement le fait qu'on m'avait dit que si je
ressentais l'un de ces symptômes actuels, je devais demander une aide d'urgence
car il pouvait s'agir d'un caillot sanguin.

J'ai dû insister pour demander un test D-dimer et j'ai reçu un ultrason très
rapide de ma jambe.

Ces tests étant clairs, le Dr X m'a dit que le caillot était probablement
décomposé dans mon système, que je devais arrêter tous les médicaments parce que
mon diagnostic actuel de péricardite était incorrect et que je n'avais
rien. J'ai été renvoyée chez moi avec un nouveau diagnostic d'“anxiété”. Je me
sentais très déstabilisée, sans réponses et confuse quant à la façon dont je
pouvais obtenir autant d'opinions différentes… Je ne savais pas qui croire !

J'ai vécu l'expérience la plus humiliante qui soit lorsqu'on m'a dit que j'étais
simplement anxieuse et que mes symptômes étaient dans ma tête. Ma question à ce
moment-là était la suivante : si elle était si sûre que l'anxiété était la cause
de ces symptômes horribles, pourquoi ne m'a-t-elle pas aidée à résoudre ce
problème ? Le Dr X m'a ensuite dit de quitter l'hôpital alors que je souffrais.

Je ne suis pas la première personne à vivre cette même expérience.

Le lendemain, lors d'un rendez-vous avec mon médecin généraliste, il a de
nouveau confirmé que mes résultats d'ECG montraient des lectures différentes
pour chaque test, ce qui indiquait une péricardite aiguë. Il a confirmé que de
telles réactions sont inexplicables et qu'elles surviennent chez des personnes
en bonne santé. Je n'ai pas eu de chance et mon traitement est incertain car il
s'agit d'un nouveau produit avec lequel ils doivent faire avec. Une réponse
honnête de la part d'un médecin signifie beaucoup plus que de se faire dire que
vous souffrez d'anxiété parce qu'ils ne connaissent pas les réponses. Le Dr X
n'a examiné que mes résultats à ce moment-là et n'a pas comparé avec les ECG
précédents. Le Dr X ne voulait tout simplement pas s'occuper de ma douleur parce
qu'elle ne connaissait pas la réponse.

Après une semaine d'épreuves, 3 visites à l'hôpital, 2 appels à l'ambulance et
un appel à Healthline, je n'ai toujours pas été prise en charge pour un suivi,
même avec des signes de coagulation du sang !

Je suis extrêmement reconnaissante envers mon médecin généraliste et le service
d'ambulance qui m'a écoutée, m'a crue et m'a rassurée alors que tous les autres
systèmes avaient échoué.

Je suis consternée par le système de santé et le gouvernement néo-zélandais. Ils
restreignent nos vies jusqu'à ce que nous recevions ce vaccin, et lorsque nous
le recevons et que cela se passe mal, la plupart d'entre eux ne sont pas
disposés à écouter ou à aider. J'ai fait ce qu'on m'a demandé de faire pour me
protéger et protéger les autres, et quand ça a mal tourné, le système m'a
énormément déçu.

J'ai encore des douleurs dans la poitrine et un gonflement du genou, mais avec
des médicaments et du repos, je me remets très lentement.

Je ne suis PAS “anti-vax”, mais j'ai personnellement choisi d'exprimer mon
expérience pour être transparente avec tout le monde, car je pense que cela fait
cruellement défaut et provoque une immense incertitude autour des vaccins.

Partager mon histoire, car l'anxiété n'est pas le problème.
