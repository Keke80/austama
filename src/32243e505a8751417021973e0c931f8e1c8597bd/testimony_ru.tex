Кэти, жена Чарльза, рассказывает его историю:

Утром с Чарльзом было все в порядке, и он пошел купить что-нибудь из еды. Он
сказал мне, что чувствует себя немного «отключённым», и спросил, могу ли я
водить машину. Я проехала, наверно, меньше километра, когда он подскочил (у меня
муж мужчина и меня ничего не пугало). Я спросила его, всё ли в порядке, и она
сказала мне: «Это меня напугало!», Я спросила его: «Что тебя напугало?» и он
ответил: «Слишком много объяснять». Это обеспокоило меня, но я продолжала вести
машину. Когда мы подъехали к ресторану. Он снова подпрыгнул, и я поняла, что
что-то пошло не так.

Я развернула машину и направилась прямо в отделение скорой помощи, которое
находилось на той же парковке. Я выбежала и подбежала к его стороне машины,
попыталась вытащить его, но у него начались галлюцинации, и он не реагировал,
так как был погружён в них. Тогда я сама попыталась вытащить его, и в итоге мне
удалось стащить его на землю между дверью и засыпанной соломой разделительной
полосой на одном из парковочных мест.

Я вбежала внутрь и закричала о помощи. Со мной тут же выбежали две медсестры и
попытались стащить его с насыпи. Прошла минута или две, пока не прибыл медик и
они смогли поместить его на ровную площадку, где они могли с ним работать.

В то время он не был в ясном сознании. Я была вне себя, но я не из тех, кто
паникует, поэтому я старалась оставаться максимально спокойной. Они пытались
оценить клинические признаки, но не знали, что случилось. Я разговаривала по
телефону, пытаясь дозвониться до друга.

Медсестры стали говорить: «Вставай, Чарльз!» Я подбежала и тоже попыталась его
поднять. У него шла слюна и он храпел. Что-то глубоко внутри меня подсказывало,
что этого не произойдет. Скорая помощь, по-видимому, вызвали 911, потому что
подъехала машина . Они начали с того, что попытались «стабилизировать» его,
сделали сердечно-легочную реанимацию и прочее. В тот момент я ясно помню, как
посмотрела на него на дорожном полотне и сказала себе: «Он выглядит мертвым». Я
была поражена!!

Парамедики продолжали работать и сказали мне, что они должны будут отвезти его в
больницу, как только его состояние стабилизируется. Через несколько минут ко мне
подошел один из санитаров и сказал: «Я хотел сказать, что я горжусь вами! Вы
сделали всё от вас зависящее: привезли его сюда и всё остальное». Я поняла, что
он уходит, и они не хотели, чтобы я чувствовала себя виноватой..

Я спросила их, как он себя чувствует, и они ответили, что всё ещё пытаются его
оживить, что дало мне ложную надежду. В конце концов, они сказали мне, что везут
его в отделение скорой помощи, и я снова подумала, что есть какая-то надежда,
так как ранее они сказали, что не примут его, пока он не станет стабильным. В
это время в приемное отделение приехали наши близкие друзья, а это, в основном,
мама, сестра и племянница моего мужа.

Когда мы приехали, нас проводили в зал ожидания и попросили подождать. Приехал
еще один близкий друг. Мы сидели там и ждали в надежде. Они наконец, пришли и
сказали, что его поместили в реанимацию и что с ним врач. У нас опять ложные
надежды.

В какой-то момент вошла доктор и спросила, кто его жена. Я представилась, и врач
спросила меня, что именно произошло. Я рассказала ей всю историю и подумала, что
она ставит ему диагноз. Это был не тот случай. Она пришла сказать мне, что
реанимировать его не удалось и что он умер.

Мы все разрыдались. Прибыл помощник судмедэксперта и спросил, не хотим ли мы
провести вскрытие. Его ассистент сказал, что посмотрит, сможет ли он это
сделать, но большинство округов Джорджии поделили экспертов, и нам нужно
заказывать вскрытие. Он сказал мне, что даст мне знать.

На следующий день ассистент судебно-медицинского эксперта позвонил моим друзьям
(они дали ему свой номер телефона) и сообщил, что мой муж перенес «гипоксическое
состояние». Он сказал, что они не собираются проводить вскрытие, потому что уже
известно, что его убило. Это вот это. Когда я получила свидетельство о смерти из
больницы, в документе было указано, что причиной смерти было «гипертоническое
сердечно-сосудистое заболевание». Я так и не получила четкого ответа о том, что
же убило моего мужа.

Мой муж был всем моим миром. Он умер за 2 дня до нашей 30-й годовщины свадьбы, и
я уже никогда не буду прежней. Потеря его и овдовение в 51 год опустошили
меня. Я сожалею о том, что он сделал прививку, потому что в глубине души знаю,
что это был как минимум фактор, который убил его.

До этого события, он был в полном порядке. Он был невероятно сильным человеком:
люди буквально называли его «землетрясением», потому что он был большой и
сильный, и я не могу согласиться с тем, что это было какое-то
сердечно-сосудистое заболевание, тем более что у него никогда не было никаких
проблем с сердцем.

Я чувствую, что моё будущее растаяло на моих глазах. Я потеряла моего лучшего
друга и мою скалу. Я не работаю уже 3 месяца. Я никогда не была без мужа с 19
лет, и я так и не повзрослела без него.

Я злюсь, но не на Бога… Мне действительно некуда направить мою обиду. Я
разбита. Я боюсь делать прививки, потому что знаю, почему я овдовела в 51 год. Я
не знаю, что ещё сказать, кроме того, что я сломлена, одинока и опустошена.
