Charles’ wife Kathy telling his story:

Charles was fine in the morning and we were going to get some food. He said that
he was feeling a little “off” and asked me if I would drive. I drove less than a
mile when he flinched (my husband was a man’s man and NOTHING scared him). I
asked him if he was okay, and he said, “It scared me!” I said, “What scared
you?” and he said, “It’s just a lot to process.” I was concerned, but kept
driving, as we were turning into the restaurant. Then he started flinching
again, and I could tell something was very wrong.

I got back in the car and drove immediately to the Urgent Care that was
basically in the same parking lot. I ran around to his side of the car and tried
to get him out to go inside, but he was hallucinating and not responding to me,
as he was preoccupied with the hallucinations. I tried to pull him out of the
car and wound up basically pulling him to the ground between the door and the
median that was covered with pinestraw in the parking lot.

I ran inside and screamed for assistance. Two female nurses came out with me
immediately and tried to get him onto the median. One or two minutes went by and
finally a male nurse came out and was able to get him to the median where they
could work on him.

By this time, he was not really lucid at all. I was beside myself, but I’m not
much of a panicker, so stayed as calm as possible. They started trying to
evaluate him, but couldn’t determine what was wrong. I was on the phone trying
to get a friend to come help me.

The nurses started saying “wake up, Charles!” and so I ran over and tried to
wake him up too. He was drooling and snoring. Something in my heart told me that
he wasn’t going to make it. The Urgent Care had apparently called 911 when I
came in, because an ambulance pulled up. They EMTs started trying to “stabilize”
him, and did CPR and other things. At one point, I remember clearly looking down
at him on the pinestraw island and thinking to myself, “he looks dead.” I was
devastated!!!

The paramedics kept working on him and said that they would leave for the
hospital once he was stabilized. A few minutes later, one of the paramedics came
up to me and said, “I want you to know I’m really proud of you! You did
everything right, getting him here and all.” I knew then that he had passed, and
they didn’t want me to feel as though it was my fault.  I asked how he was, and
they said they were still working on him – which gave me false hope. Finally,
they said they were taking him to the ER, so again, I thought there was hope
since they had previously said they wouldn’t leave until he was stable. By this
time, our close friends, who are essentially my husband’s mother, sister and
niece had arrived, and we all went immediately to the ER.

When we got there, we were escorted into a family waiting room and told to
wait. Another close friend arrived. We sat on pins and needles – clinging to
hope. Then they came in and said he was in Critical Care, and that a doctor was
with him. More false hope.

Eventually, the doctor came in and asked who his wife was. I identified myself,
and the doctor asked exactly what had happened. I told her, and thought that she
was trying to diagnose him. She wasn’t. She then told me that they had never
been able to revive him, and that he had passed away.

We all burst into tears! Eventually, the assistant coroner arrived, and we
requested an autopsy. The assistant coroner said that he would see if he could
get one performed, but that most of the counties in GA shared one medical
examiner, and we’d have to request an autopsy. He said he’d let me know.

The next day, the assistant coroner called my friends (they gave him their
number) and said that my husband had suffered a “hypoxic event.” He said they
were not going to do an autopsy, because they knew what had killed him. That was
that. When I got the death certificate from the hospital, it said the cause of
death was “Hypertensive Cardiovascular Disease.” So I never got a clear answer
on what killed my husband.

My husband was my world. He died 2 days shy of our 30th Wedding Anniversary, and
I will never be the same again. Losing him, and becoming a widow at 51 is
devastating. I have regretted his getting the vaccine because I know in my heart
that it was AT LEAST a contributing factor, and most likely what killed him.

He was totally fine prior to this event. He was an incredibly strong man –
literally, people called him “Earthquake” because he was so strong – and I can’t
accept that this was any kind of cardiovascular disease when he never showed any
signs of heart issues at all.

I feel like my future has been dissolved before my very eyes. I lost my best
friend, and my rock. I didn’t work for three months. I had been with my husband
since I was 19 years old and never really “adulted” without him.

I am angry, though not at God… I don’t really have a place to direct my anger. I
am broken. I am terrified of mandates that say we must get vaccinated because I
know that’s why I am a widow at 51. I don’t know what to say other than I am
broken, lonely and devastated.

